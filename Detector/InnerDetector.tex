\subsection{Inner Detector}
The inner detector (ID) provides full tracking information for $|\eta|<2.5$. 
The purpose of the ATLAS tracking detectors is to make high precision measurements of charged particle tracks near the beam line. 
These measurements are used for primary and secondary vertex finding and momentum determination. 
The tracking detectors have to be able to deal with the high track multiplicity expected at the design luminosity of the LHC.

\begin{figure}  
\centering  
\includegraphics[width=0.9\textwidth]{figures/Detector/AtlasID.jpg}
\caption[The ATLAS Inner Detector]{
A schematic of the ATLAS Inner Detector.
Figure from \cite{ref:ATLASExp}.
  \label{Det:ATLASID}
}
\end{figure}

Figure \ref{Det:ATLASID} shows the different components of the ATLAS inner detector that will be discussed. 
Both the pixel detector and the semiconducting tracker have a tracking region out to $|\eta|<2.5$, and the transition radiation tracker goes out to $|\eta|<2$. 

The pixel detector is the closest to the interaction point, and thus has the highest granularity of the inner detector trackers. 
There are three pixel layers in the barrel region each having a 2d segmentation, with a minimum size of $\phi\times z$ of $50\times400~\mu\rm{m}^2$, giving a well measured space point. 
The main use of the pixel detector is to find B hadrons and $\tau$ leptons. 

Further from the interaction point is the semi conductor tracker (SCT) which is less granulated. 
%*how does it work? 8 measurement from 4 layers + some detector pairs *. 
The main use of the SCT is determining track momenta, impact parameters and vertex positions.

Furthest away from the interaction point is the transition radiation tubes detector (TRT), which provides a large number of hits per track. 
The position accuracy of these hits is less accurate than that from the pixel and SCT detectors
The TRT was not used for tracking in 2010 data-taking.
