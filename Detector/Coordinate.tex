\subsection{ATLAS coordinate system}
\label{sec:Det:Coord}
The ATLAS experiment uses a right-handed Cartesian coordinate system. 
The beam line defines the z-axis, with the x-y plane being perpendicular to this. 
The positive x direction is defined going from the interaction point to the centre of the LHC ring. 
The positive y direction points upwards from the interaction point. 
The detector side in the positive z direction is called A and the side in the negative z direction C.
The azimuthal angle, $\phi$, is defined as the angle in the x-y plane around the beam, and the polar angle, $\theta$, is the angle to the beam line.
Using the polar angle, pseudorapidity, $\eta$ is defined by 
\begin{equation}
\eta = - \ln \tan \left(\frac{\theta}{2}\right).
\label{Detector:eta}
\end{equation}
The angle between two objects, $\Delta R$, is defined by 
\begin{equation}
\Delta R = \sqrt{(\Delta \eta)^2+(\Delta \phi)^2} .
\label{Detector:dR}
\end{equation}

All transverse variables (such as transverse momentum, $p_T$, and transverse energy, $E_T$) use only the x and y components of the variable and so are defined in the x-y plane.
