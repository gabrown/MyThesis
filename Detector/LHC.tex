\section{The Large Hadron Collider}
\label{sec:Det:LHC}

The Large Hadron Collider (LHC) is situated 100 m below the border between Switzerland and France near the Swiss city of Geneva. 
The LHC was designed to provide two proton beams with 2808 bunches in each beam colliding with 25 ns bunch spacing at a centre-of-mass energy of 14 TeV.
Around the LHC ring, which is 26.6 km long, there are four interaction points. 
The four experiments at these interaction points are ATLAS, CMS, LHCb and ALICE.
ATLAS and CMS are general purpose detectors designed to be able to detect a broad range of physics processes. 
ALICE is designed to investigate heavy ion collisions, and LHCb is designed to explore CP violation and rare B-decays.
Figure \ref{Det:LHC} shows the layout of the LHC, and the four experiments.

\begin{figure}
\centering
\includegraphics[width=0.9\textwidth]{figures/Detector/AtlasSite.jpg}
  \caption[The Large Hadron Collider complex]{
The Large Hadron Collider and the sites of the 4 main LHC experiments.
\label{Det:LHC}
}
\end{figure}

A series of accelerators provide proton bunches to the LHC at an energy of 450 GeV.
The proton bunches are then accelerated around the LHC by an array of superconducting dipole magnets to provide a beam energy up to 7 TeV.
The beam energy during the 2010 and 2011 proton-proton collisions was 3.5 TeV, in 2012 it was 4 TeV.


\subsection{Luminosity}

The event rate for $pp \rightarrow X$ is
\begin{equation}
R_X = L \sigma_X,
\label{Det:Lumi}
\end{equation}
where $\sigma_X$ is the cross-section for the process and $L$ is the instantaneous luminosity.
Achieving a large integrated luminosity,
\begin{equation}
{\cal L} = \int L\cdot dt,
\label{Det:IntLumi}
\end{equation}
is important for observing rare physics processes that have a low cross-section. 
Accurate luminosity determination is important for measuring differential cross-sections of processes from the event rate. 

Instantaneous luminosity is defined by,
\begin{equation}
L=\frac{N_b^2n_bf_{rev}\gamma_r}{4\pi\epsilon_n\beta^\star}F,
\label{Det:Lumi}
\end{equation}
where $N_b$ is the number of particles per bunch, $n_b$ is the number of colliding bunches per beam, $f_{rev}$ is the revolution frequency, $\gamma_r$ is the relativistic gamma factor, $\epsilon_n$ is the normalised transverse beam emittance, $\beta^\star$ is the $\beta$ function at the interaction point, and $F$ is the geometric luminosity reduction factor due to the crossing angle of the beams.


Equation \ref{Det:Lumi} is useful to understand both the luminosity and how to increase it. 
Experimentally, the instantaneous luminosity can be determined by measuring the interaction rate in various ATLAS sub-detectors.
The luminosity defined by the interaction rate is,
\begin{equation}
L=\frac{\mu n_bf_{rev}}{\sigma_{inel}}=\frac{\mu_{vis}n_bf_{rev}}{\sigma_{vis}}
\label{Det:Lumi2}
\end{equation}
where $\mu$ is the number of inelastic collisions per bunch crossing, $\sigma_{inel}$ is the inelastic cross-section, $\mu_{vis}$ is the number of visible inelastic collisions per bunch crossing, and $\sigma_{vis}$ is a calibration constant related to the visible inelastic cross-section. 
This is obtained in special runs using Van der Meer scans and provide a luminosity uncertainty of 3.4\% \cite{ref:Lumi}.

\subsubsection{2010 run}
The 7 TeV centre-of-mass proton-proton run in 2010 was the first substantial data-taking period provided by the LHC. 
The initial runs provided peak luminosity of \lumi{\approx0.01\times10^{30}} from one pair of interacting bunches with a very low number of protons per bunch. 
By the end of the 2010 data-taking run, the peak luminosity was \lumi{\approx2\times10^{32}} from 348 colliding bunches.
The luminosity increase was mainly achieved by increasing the number of colliding bunches per beam and the number of protons per bunch, though decreasing the $\beta^\star$ and reducing the beams crossing angle also increased the luminosity.

The data is split into luminosity blocks, runs and periods.
A luminosity block corresponds to the luminosity information stored in two minute intervals during a run. 
A data run is a group of luminosity blocks consecutively taken. 
A data period is a group of data runs with similar beam parameter conditions.

The 2010 data taking run was split into nine different data periods, which corresponded to a total integrated luminosity delivered of 48.1 $\rm pb^{-1}$.
Table \ref{Det:Periods} shows the data periods for the 2010 LHC run, and how the beam conditions and peak luminosity changed.  
In addition, periods G-I used bunch trains where bunches are grouped together with 150 ns spacing between the bunches.  
As instantaneous luminosity increased, the level of ``pile-up'', which is multiple proton-proton interactions in the same bunch crossing, increased.

\begin{table}
\begin{center}
\begin{tabular}{|c|c|c|c|c|}
\hline
Period&Date&Peak Luminosity& $N_b$ & $n_b$ \\
& &\lumi{}& $\times10^{11}$ & \\
\hline
A & Mar 30 - Apr 18 & 0.004 & $<0.01$ & 1 \\
B & Apr 23 - May 17 & 0.06 & 0.01-0.6 & 1-3 \\
C & May 18 - June 5 & 0.2 & 0.7-1.9 & 3-8 \\
D & June 18 - July 19 & 1.6 & 3-8 & 2-8 \\
E & July 29 - Aug 18 & 3.9 & 12-14 & 16 \\
F & Aug 19 - Aug 30 & 10 & 35-100 & 32-36 \\
G & Sept 22 - Oct 7 & 70 & 100-200 & 50-186 \\
H & Oct 8 - Oct 18 & 150 & 250-350 & 233-300 \\
I & Oct 24 - Oct 29 & 210 & 350-400 & 300-350 \\
\hline
\end{tabular}
\caption[2010 Data Period Information]{
   Beam information for the different data taking periods in 2010 data.
\label{Det:Periods}
}
\end{center}
\end{table}

