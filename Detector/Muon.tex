\subsubsection{Muon Detectors}

Furthest from the interaction point are the muon detectors. 
The muon detectors measure the hits from muons which are bent by the magnetic field from the barrel toroid ($|\eta|<1.4$) and end cap toroids ($1.6<|\eta|<2.7$), and a combination of both in the region between. 
The muon system has seperate dedicated detectors for both precision position measurements and for triggering on muons.

For the presision measurement, Monitored Drift Tubes (MDT) are used in the rapidity range $|\eta|<2.7$, with the higher granularity Cathode Strip Chambers (CSC) used in the more forward region of $2<|\eta|<2.7$ in the innermost layer. 

The muon triggering system consists of Resistive Plate Chambers (RPC) in the barrel region and Thin Gap Chambers (TGC) in the end cap region. 
The muon triggering system covers the region of $|\eta|<2.4$. 


