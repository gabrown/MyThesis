The Large Hadron Collider provided proton-proton colisions at a world record centre-of-mass energy of 7 TeV.
The ATLAS detector aims to descover new physics coming from these collisions.
**** A line about why there are always jets****
New physics signals are often generated in association with jets.
Analysis cuts and event selection on this additional radiation can force the measurement into extreme regions of phase space.
Studying dijet production with additional radiation cuts, the extreme phase space can be probed. 
These studies are important precision QCD measurements which can be compared to various event generators.



In this thesis the standard model of particle physics is reviewed with particular emphasis on Quantum Chromo Dynamics in Chapter \ref{chp:Theory}. 
In Chapter \ref{chp:LHC_ATLAS} the LHC experiment and ATLAS detector are discribed. 
Monitoring of the calorimter systems are discribed in Chapter \ref{chp:HLTCalo}.
In-situ methods used to assess the effectiveness of jet calibration and forward jet properties are presented in Chapter \ref{chp:JetPerf}.
Chapter \ref{chp:GBJ1} presents the first ATLAS paper on dijet production with a veto on jets bounded by the dijet system as a function of the dijet rapidity seperation and the average transverse momentum. 
In Chapter \ref{chp:GBJ2} the analysis described in Chapter \ref{chp:GBJ1} is extended to study dijets with large rapidity seperation, and a measurement of the azimuthal decorraltion is made.
Finally, in Chapter \ref{chp:Conc} the results from the previous Chapters are summarised.



