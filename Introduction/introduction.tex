ATLAS is a multipurpose detector built with the aim of measuring a wider variety of physics processes, which occur in the high energy proton-proton collisions produced by the Large Hadron Collider (LHC) at CERN.
The majority of these measurements are influenced by the presence of jets.
Examples include physics signals with jet final states or jets produced in association with other particles.
Stringent analysis cuts on additional jets used, for example, in the search for new physics, can force the event topology into regions of phase space that are difficult to calculate in fixed-order perturbation theory.
This means the theory models used to predict backgrounds may be imprecise.
It is important to test model predictions with precise measurements in similar phase space regions.

Dijet production is ideal for testing the model predictions due to the large number of events even in the phase space near the kinematic limit. 
Dijet production with a large average jet transverse momentum or large rapidity separation reflect the cuts used in heavy resonance searches and vector boson production of the Higgs boson. 
Precision measurements of observables sensitive to higher order Quantum Chromo Dynamics (QCD) emission can test the appropriate models.

The fraction of dijet events which survive a jet veto on the jet activity between the dijet system is studied as a function of the rapidity separation of the dijet system, the mean transverse momentum of the dijet system and the jet veto scale for separations up to six units of rapidity in the range of mean transverse momentum $50<\ptb{}<500$ GeV.

In Chapter \ref{chp:Theory} of this thesis the Standard Model (SM) of particle physics is reviewed with particular emphasis on the relevant aspects of QCD and the analysis observables are defined and discussed. 
In Chapter \ref{chp:LHC_ATLAS} both the LHC experiment and the ATLAS detector are described. 
Monitoring of the calorimeter systems is described in Chapter \ref{chp:HLTCalo}.
In-situ methods used to assess the effectiveness of jet calibration and forward jet properties are presented in Chapter \ref{chp:JetPerf}.
Chapter \ref{chp:GBJ1} presents the analysis which formed the basis of the first ATLAS paper \cite{ref:ATLASGap} on dijet production with a veto on jets bounded by the dijet system, as a function of the dijet rapidity separation and the average transverse momentum. 
In Chapter \ref{chp:GBJ2} the analysis is extended to study dijets with a large rapidity separation, and a measurement of the azimuthal decorrelation is made.
This analysis is currently in the process of internal review by the ATLAS collaboration \cite{ref:GBJInternal}.
A summary and conclusion of the thesis is given in Chapter \ref{chp:Conc}.


