
The Large Hadron Collider provided proton-proton collisions at a centre-of-mass energy of 7 TeV.
ATLAS is a multipurpose detector built with the purpose of measuring a wider variaty of physics processes.
The majority of measurements at a hadron-hadron collider are influenced by the presence of jets.
Examples include phyics signals with jet final states or jets produced in association with other particles.
Stringent analysis cuts on additional jets, used, for example, in the search for new physics, can force the event topology in to regions of phase space that are difficult to calculate in fixed order pertubation theory.
This means the theory models used to predict backgrounds may be imprecise.
It is important to test model predictions in precise mesurements on similar phase space regions.

Dijet production is ideal for testing the model predictions due to the large number of events even in phase space near the kinimatic limit. 
Dijet production with a large average transverse momentum or large rapidity separation reflect the cuts used in heavy resonance searches and vector boson production of the Higgs boson. 
Precision measurements of observables sensitive to higher order QCD emission can test the appropriate models.

The fraction of dijet events which survive a jet veto on the jet activity between the dijet system is studied as a function of the rapidity separation of the dijet, the mean transverse momentum of the dijet and the jet veto scale for separations upto six units of rapidity and mean transverse momentum of $50<\ptb{}<500$ GeV.

In Chapter \ref{chp:Theory} of this thesis the Standard Model (SM) of particle physics is reviewed with particular emphasis on the relevent aspects of Quantum Chromo Dynamics (QCD) and the analysis observables are defined and discussed. 
In Chapter \ref{chp:LHC_ATLAS} the LHC experiment and ATLAS detector are described. 
Monitoring of the calorimeter systems is described in Chapter \ref{chp:HLTCalo}.
In-situ methods used to assess the effectiveness of jet calibration and forward jet properties are presented in Chapter \ref{chp:JetPerf}.
Chapter \ref{chp:GBJ1} presents the analysis which formed the basis of the first ATLAS paper on dijet production with a veto on jets bounded by the dijet system as a function of the dijet rapidity separation and the average transverse momentum. 
In Chapter \ref{chp:GBJ2} the analysis described in Chapter \ref{chp:GBJ1} is extended to study dijets with a large rapidity separation, and a measurement of the azimuthal decorralation is made.



