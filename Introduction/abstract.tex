This thesis describes the measurements of jet activity in the rapidity region between a dijet system formed in proton-proton collisions at a centre-of-mass energy of 7 TeV.
The data used were collected by the ATLAS detector during 2010 at the Large Hadron Collider at CERN.
A number of observables that probe additional quark and gluon radiation in the dijet topology were studied.

The development and performance of the monitoring system for the ATLAS calorimeter high level trigger is described.
The performance of the jet calibration and a study of the properties of jets in the forward calorimeter is also given.

The fraction of events that survive a veto on jets with transverse momentum above a jet veto scale, \qz{}, in the rapidity region between the dijet system is measured for dijets with mean transverse momentum $50<\ptb{}<500$ GeV and rapidity separation, \dy{}, of up to six.
The mean number of jets that have a transverse momentum above the jet veto scale in the rapidity region between the dijet system is also measured. 
These measurements are compared to state of the art theoretical calculations from HEJ and POWHEG, and also compared to PYTHIA, HERWIG++ and ALPGEN Monte Carlo generators.

The results of a preliminary analysis of dijet events with a large rapidity seperation are given.
In this analysis azimuthal decorrelation variables have also been measured.
