\section{Overview of Other Analysis Components}
\label{sec:GBJ1:OtherWork}
To compare the data to theoretical models further aspects of the analysis were performed by other members of the analysis team. 
This section will review the unfolding of the data to hadron level and the combination of the main systematic uncertainties in this analysis will then be discussed.

Unfolding is required to produce measurements that are independent of detector effect.
This allows the corrected data to be compared to generators and theoretical predictions which do not simulate the ATLAS detector.§
This was done using a bin-by-bin method comparing the final distributions from the PYTHIA sample at hadron and detector level.
The eventual correction factors were typically less than $10\%$.


The systematic uncertainties from various sources were considered by the other analysers. 
They were:
\begin{description}
\item[Jet Energy Scale (JES) uncertainty] varies depending on the $\Delta y$ bin. 
It can be as large at 5-10\% and is a significant source of uncertainty.

\item[Unfolding uncertainty] comes mainly from the limited statistics of the PYTHIA sample used to obtain the unfolding factors. 
It is a significant source of uncertainty with values as large as 5\%.

\item[Cosmic and beam backgrounds] event rates were estimated using triggers during non-filled bunch crossings. 
The rate of such events was found to be very small compared to the rate of signal events, and the effect is taken to be negligible.

\item[Trigger Bias] was estimated by comparing the gap fraction and jet multiplicity distributions with the analysis trigger strategy to the distributions obtained using the minimum bias stream and was found to be negligible in all regions.

\end{description}


