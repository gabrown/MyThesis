\section{Topology Selection}
\label{sec:GBJ1:AnalSel}
The jets used in this analysis were reconstructed using the anti$-k_T$ algorithm with a radius parameter R=0.6\footnote{The anti$-k_T$ algorithm is described in Section XXX}. 
Only jets that have \pt{}$>20$ GeV and $|y|<4.4$ have been used, as these are the regions that have a well defined jet energy scale and jet cleaning cuts. 


The analysis uses two different dijet selection criteria, ``Leading \pt{} Dijet Selection" and ``Forward/Backward Selection", to define two boundary jets. 

In the leading dijet \pt{} selection the boundary jets are the two highest \pt{} jets in the event, and in the forward/backward selection the two boundary jets are the most forward (positive rapidity) and most backward (negative rapidity) jet in the event. 



Once the boundary jets have been assigned, an additional cut is applied to the average transverse momentum, \ptb{},of the boundary jets of 50 GeV. 
The \ptb{} cut is required to ensure that the event is in a high efficiency trigger region. 
These cuts define the inclusive events for the analysis. 


To study the activity in the rapidity region between the boundary jets, two variables are investigated. 
The first is the ``Gap Fraction". This is the fraction of the inclusive sample  that passes a jet veto, which excludes events that have an additional jet with \pt{} greater than the jet veto scale,\qz{}, in the dijets rapidity region. 
The gap fraction, \gap, is defined by,
\begin{equation}
f_{\rm gap} = \frac{n (\qz{})}{N},
\label{GBJ1:fgap}
\end{equation}
where n(\qz{}) is the number of gap events, events which pass the jet veto, and $N$ is the inclusive number of events.
The nominal choice of the jet veto scale is \qz{}$=20$ GeV. 

The second variable studied is the mean number of jets, which have \pt{}$>$\qz{}, in the dijet rapidity region, ie 
\begin{equation}
\nb{} = 
\label{GBJ1:fgap}
\end{equation}
where ....

*Now discuss some errors*
