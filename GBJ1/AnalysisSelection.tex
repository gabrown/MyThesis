\section{Topology Selection}
\label{sec:GBJ1:AnalSel}

The jets used in this analysis were reconstructed using the \antikt{} algorithm with a radius parameter R=0.6, as described in Section \ref{sec:Theory:Jets}.
Only jets that have \pt{}$>20$ GeV and $|y|<4.4$ have been used, as these are the regions that have a well defined jet energy scale and jet cleaning cuts, as discussed in Section \ref{sec:Det:Jets}. 

The analysis uses two different dijet selection criteria, ``Leading \pt{} Dijet Selection'' and ``Forward/Backward Selection'', to define two boundary jets. 
In the leading dijet \pt{} selection the boundary jets are the two highest \pt{} jets in the event, whereas in the forward/backward selection the two boundary jets are the most forward (positive rapidity) and most backward (negative rapidity) jets in the event. 
Once the boundary jets have been defined, an additional cut is applied to the average transverse momentum, \ptb{}, of the boundary jets of 50 GeV. 
This ensures that the dijets are in a high efficiency trigger region, which were studied in early plausibility studies \cite{ref:GBJConf1} by other members of the analysis team. 
These cuts define the inclusive events for the analysis. 


Two variables are investigated; the gap fraction and the mean  number of jets in the rapidity interval between the boundary jets. 
The gap fraction, \gap, defined in Equation \ref{Theory:GapFraction} can be measured by,
\begin{equation}
f_{\rm gap}(\qz{}) = \frac{\sigma_{0}}{\sigma} =  \frac{N (\qz{})}{N},
\label{GBJ1:fgap}
\end{equation}
where N(\qz{}) is the number of events that do not contain a jet with $\pt{}>\qz{}$ in the rapidity interval between the boundary jets, and $N$ is the number of events in the inclusive sample, if the trigger, acceptance and luminosity biases cancel in the ratio.
The nominal choice of the jet veto scale is \qz{}$=20$ GeV. 
The mean number of jets in the rapidity interval is defined for jets with $\pt{}>20$ GeV. 

Both the gap fraction and the mean number of jets are measured as a function of dijet rapidity region, \dy{}, for multiple slices in \ptb{}, and also as a function of \ptb{} for multiple slices in \dy{}. 
The gap fraction is also measured as a fraction of the veto scale, \qz{}
