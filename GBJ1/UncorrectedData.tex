% Why is comparing to uncorrected data important?

\section{Data compared to reconstructed PYTHIA }
\label{sec:GBJ1:Uncorr}
This section presents the data compared to the reconstructed PYTHIA sample, to check that the PYTHIA sample approximately agrees with data so that it can be used for unfolding the detecter effects in the data. 

Figures \ref{UncorrIncl_dy} - \ref{Uncorr_Pt3_dy} are various distributions with data and the reconstructed PYTHIA sample shown, and also the ratio between them.
Figure \ref{UncorrIncl_dy} shows the inclusive $\Delta y$ distribution for three different $\bar{p_T}$ slices. Figure \ref{Uncorr_Pt3_dy} shows the inclusive $p_T^{veto}$ distribution for two different slices in $\Delta y$ and $\bar{p_T}$ space. Figure \ref{Uncorr_GF} shows slices of the gap fraction against $\Delta y$ and $\bar{p_T}$.

The three figures show good agreement between the data and the reconstruted PYTHIA sample, and give confidence that the PYTHIA sample can be used for the unfolding factors. 

\begin{figure}
\centering
\mbox{
              \subfigure[]{\epsfig{figure=figures/GBJ1/UncorrectedData/Inclusive_selA_Ave_pT_70_90_Norm.eps,width=0.5\textwidth,height = 6cm}}\quad
              \subfigure[]{\epsfig{figure=figures/GBJ1/UncorrectedData/Inclusive_selA_Ave_pT_90_120_Norm.eps,width=0.5\textwidth,height = 6cm}}\quad
}
\mbox{
              \subfigure[]{\epsfig{figure=figures/GBJ1/UncorrectedData/Inclusive_selA_Ave_pT_180_210_Norm.eps,width=0.5\textwidth,height = 6cm}}\quad
                              }
\caption[Comparison between data and PYTHIA sample in the inclusive distribution for $\Delta y$]{
Fraction of events for each $\Delta y$ bin for 2010 uncorrected data and PYTHIA sample. Shown are (a) $70<\bar{p_T}<90$, (b) $90<\bar{p_T}<120$ and (c) $210<\bar{p_T}<240$ slices.
\label{UncorrIncl_dy}}
\end{figure}

\begin{figure}
\centering
\mbox{
              \subfigure[]{\epsfig{figure=figures/GBJ1/UncorrectedData/Pt3_dY_2_3_pt_90_120_PYTHIA.eps,width=0.5\textwidth,height = 6cm}}\quad
              \subfigure[]{\epsfig{figure=figures/GBJ1/UncorrectedData/Pt3_dY_2_3_pt_180_210_PYTHIA.eps,width=0.5\textwidth,height = 6cm}}\quad
                              }
\caption[Comparison between data and PYTHIA sample for Pt3 distribution]{Faction of events for each $p_T^{veto}$ bin for 2010 uncorrected data and PYTHIA sample. Shown are (a) $90<\bar{p_T}<120$ and $2 < \Delta y <3$, and (b) $90<\bar{p_T}<120$ and $2 < \Delta y <3$ slices.\label{Uncorr_Pt3_dy}}
\end{figure}

\begin{figure}
\centering
\mbox{
              \subfigure[]{\epsfig{figure=figures/GBJ1/UncorrectedData/RatioGF_selA_Ave_pT_90_120.eps,width=0.5\textwidth,height = 6cm}}\quad
              \subfigure[]{\epsfig{figure=figures/GBJ1/UncorrectedData/RatioGF_selA_DeltaY_2_3.eps,width=0.5\textwidth,height = 6cm}}\quad
                              }
\caption[Comparison of gap fraction between data and PYTHIA sample for $\Delta y$]{Gap fraction against (a) $\Delta y$ for the $90<\bar{p_T}<120$ slice and (b) $\bar{p_T}$ for the $2 < \Delta y <3$ slice, for 2010 uncorrected data and PYTHIA sample.\label{Uncorr_GF}}
\end{figure}


