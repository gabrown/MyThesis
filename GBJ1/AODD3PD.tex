\section{Data Stability}
\label{sec:GBJ1:DataStab}

\subsection{Comparison between AOD and D3PD}
\label{sec:GBJ1:AODD3PD}

The cuts explained in section \ref{sec:GBJ1:EvtSel} were applied, and a cut flow was defined to check that there was stable running over the data and with the cuts. The analysis was done by both the author using the data in AOD format, and also independently by a different member of the analysis team running over D3PDs, which is a flat ntuple format. Table \ref{GBJ1:CutFlow} shows the number of events after each cut for both the AOD and D3PD formats, and also the fractional difference between the two, for the different data periods. The differences in the table are of the order 1:1000 which indicates that the cut flow is stable.  

   
\begin{table}
\small
\footnotesize

\begin{tabular}{|c|c|c|c|c|c|c|c|}
\hline
%
\input{figures/GBJ1/DataStability/AODD3PD/GBJCutFlow-Loose.csv}
%
\hline
\end{tabular}
\caption[Cut flow comparison between AOD and D3PD formats]{Cut flow comparison between AOD and D3PD formats for different data periods.}
\label{GBJ1:CutFlow}
\end{table}
