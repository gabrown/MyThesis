%introduction
The emission of energy into the rapidity region between a dijet system is studied for the first time at a hadron collider in this analysis.

The dijet topology is identified using two different selection criteria. First, is the two highest \pt{} jets in the event which is labeled in plots as ``Leading \pt{} Dijet Selection". Second, is the most forward and most backwards jets in the event (above a given \pt{}) which is labeled in plots as "Forward/backward selection".

%$\Delta y$ versus $\frac{\bar{p_T}}{p_T^{veto}}$ phase space (as defined in section XXX) has been probed by slicing it up into different regions (in $\Delta y$ and ${\bar{p_T}}$) and by looking how the gap fraction changes as a function of either $\Delta y$, ${\bar{p_T}}$ or ${p_T^{veto}}$.

The fraction of dijet events that do not have a third jet with \pt{}$>$\qz{} in the rapidity interval bounded by a dijet system, the gap fraction, is studied as a function of the dijet rapidity seperation, \dy{}, the average transverse momentum of the dijet, \ptb{} and the veto scale, \qz{}.
%The gap fraction as a function of $\Delta y$ is studied by having multiple slices in ${\bar{p_T}}$, as well as having a fixed ${p_T^{veto}}$ value. 
%The same was done for ${\bar{p_T}}$ but now slicing in $\Delta y$. 
%The gap fraction distributions as a function of ${p_T^{veto}}$ are sliced in both $\Delta y$ and ${\bar{p_T}}$.
The dependance of the gap fraction on \dy{}, \ptb{}, and \qz{} is studied after keeping two of the variables fixed.

The topology and event selection are outlined in Section \ref{sec:GBJ1:AnalSel} and \ref{sec:GBJ1:EvtSel}, respectively. 
In Section \ref{sec:GBJ1:DataStab}, the robustness of the event selection will be examined. 
In Section \ref{sec:GBJ1:Uncorr}, the selected data will be compared to the simulated PYTHIA sample.
Section \ref{sec:GBJ1:OtherWork} outlines the work done by other members of the analysis team that was required to get the final measurements, which are presented in Section \ref{sec:GBJ1:FinalPlots}. 

%Sources of systematic uncertainty arising from the event selection, especially from the jet cleaning, and the pile-up cuts.
 
%Then, a look into how the data agrees with reconstructed PYTHIA to check that the agreement is good enough to use PYTHIA for unfolding the detector effects

%Finally, the final results are shown, with explanations of the work done by other members of the analysis team to get these results.

%-introduction to the analysis
%   -define why we look at it
%   -what observables we want to look at
%
%-give a bit of info on what is going to be in each chapter
