\section{Final Plots}
\label{sec:GBJ1:FinalPlots}
In this section, the final plots that were shown in the publication are discussed. The plots shown are from  the leading $p_T$ dijet selection and the forward/backwards selection and show the gap fraction as a function of $\Delta y$, $\bar{p_T}$ and $Q_0$, and also the mean number of jets in the rapidity interval as a function of $\Delta y$ and $\bar{p_T}$. When $\Delta y$ and $\bar{p_T}$ is studied, $Q_0$ is always at a fixed value of 20 GeV. The data is sliced up into different regions, and so the gap fraction versus $\Delta y$ distribution is shown for seven different $\bar{p_T}$ ranges, and similarly for the $\bar{p_T}$ distribution. Each slice is offset to allow the data to be shown on one plot. The ratio of the theory predictions to data is shown next to each plot.

On each plot, the black points are the data points with statistical uncertainty bars, with the yellow bands indicating the systematic uncertainty from both the JES and unfolding. The red and blue dotted lines are the POWHEG + PYTHIA and POWHEG + HERWIG predictions respectively, and the blue band is the HEJ prediction with a theoretical uncertainty.

Figures \ref{GBJ1:dYSelA} and \ref{GBJ1:pTSelA} show the gap fraction as a function of $\Delta y$ and $\bar{p_T}$ respectively. These figures illustrate that HEJ  agrees well with the data for the complete range in $\Delta y$ for low $\bar{p_T}$, but at higher $\bar{p_T}$ slices it seems to deviate away from the data giving too many gap events. As explained in section XXX, HEJ works by resumming terms of $\Delta y$, but has no resummation when $\bar{p_T}>>Q_0$. It is, therefore not too surprising that a deviation if observed at larger $\bar{p_T}/Q_0$ ($Q_0$ is fixed). 
POWHEG + PYTHIA probably gives the best overall description of the data, but at large $\Delta y$ the gap fraction is too low. *explain why this might be for POWHEG in general*  POWHEG + HERWIG seems to give too much activity in all phase space regions, and this deviation becomes larger for larger $\Delta y$. 


%Figures \ref{GBJ1:dYSelA} and \ref{GBJ1:pTSelA} are the gap fraction as a function of $\Delta y$ and $\bar{p_T}$ respectively. Figure \ref{GBJ1:dYSelA} illistrates that for low $\bar{p_T}$ that HEJ agrees well with the data for the complete range in $\Delta y$, but at higher $\bar{p_T}$ slices it seems to deviate away from the data. The figure also shows how both POWHEG curves agree at low $\Delta y$, but deviates at higher $\Delta y$. Figure \ref{GBJ1:pTSelA} makes this point even more apparent where POWHEG + PYTHIA agrees well with the data accross the $\bar{p_T}$ range for low $\Delta y$ slices, but then deviates in the larger $\Delta y$ slices.


Figure \ref{GBJ1:Q0SelA} is the gap fraction as a function of $Q_0$ for various $\Delta y$ and $\bar{p_T}$ slices. The figure shows that none of the theoretical curves agree well with the data when the large $\Delta y$ or $\bar{p_T}$ slices are considered. Both HEJ and POWHEG + PYTHIA do well in the low $\Delta y$ and $\bar{p_T}$ slice. At larger $\Delta y$ and $\bar{p_T}$ slices they both deviate from the data, with HEJ improving its agreement at larger values of $Q_0$. As seen in the previous figures, POWHEG + HERWIG still gives a gap fraction lower than the POWHEG + PYTHIA, and that is far from the data for all ranges. 


%Figure \ref{GBJ1:Q0SelA} is the gap fraction as a function of $Q_0$ for various $\Delta y$ and $\bar{p_T}$ slices. The figure shows that the POWHEG + PYTHIA agrees with the data quite well for the low $\Delta y$ slices (even with a larger $\bar{p_T}$ slice), but seems to undershots the data for the larger $\Delta y$ slices, indicating that there is not enough activity in the gap region for large $\Delta y$ events. HEJ doesn quite well in the low $\Delta y$ and low $\bar{p_T}$ slice, but overshots at low $Q_0$ at either larger $\Delta y$ or $\bar{p_T}$ slices, indicating that events are two likely to have a jet with $p_T>20$ GeV in all events bar the low  $\Delta y$ or $\bar{p_T}$ region.


Figures \ref{GBJ1:NjetsdYSelA} and \ref{GBJ1:NjetspTSelA} are the mean number of jets in the rapidity interval between the boundary jets as a function of $\Delta y$ and $\bar{p_T}$ respectively. This is an alternative method of probing the activity between the boundary jets. The gap fraction is only concerned with the leading jet in the rapidity region, but the average number of jets considers all jet in that region with a $p_T>20$ GeV. In both plots it can be seen that, except at very low $\bar{p_T}$, HEJ underestimates the the number of jets in the gap region, which is in agreement with what is seen in the gap fraction plots. POWHEG + PYTHIA agrees well  with the data in all regions and it is still the best descriptor of the data, whereas POWHEG + HERWIG seems to slightly overestimate the data, and gets worse at lower $\bar{p_T}$.  

%Figures \ref{GBJ1:NjetsdYSelA} and \ref{GBJ1:NjetspTSelA} are the mean number of jets in the rapidity interval between the boundary jets as a function of $\Delta y$ and $\bar{p_T}$ respectively. In both plots it can be seen that, except at very low $\bar{p_T}$, HEJ underestimates the the number of jets in the gap region. POWHEG + PYTHIA does quite well in all regions of agreeing with the data, whereas POWHEG + HERWIG seems to slightly overestimate the data thorught the curves.  
 
*I intend to also talk about the Selection B stuff, but there is a lot of it, and I think I will need to discuss how much will go in, and also a chat with Andy to fully understand what we see in all the plots.*


Since publishing the results of this analysis, several papers [reference Simones and Jeppes here] have compared  their theoretical predictions to the ATLAS data. 
Reference [Jeppe] commented on the ATLAS data, coming to similar conclusions to our paper. They then went on to investigate when fixed order calculations become unreliable in this topology of event.
Reference [Simones] used an all-order resummation in $\log\frac{\bar{p_T}}{Q_0}$ terms to compare to the data. After matching to 2$\rightarrow$3 matrix element, the results agreed very well with the data in both the gap fraction versus $Q_0$ and  $\Delta y$, coming to their stated conclusion that: ``The message is clear: the accuracy of the ATLAS data already demands better theoretical calculations."

*Here do I want to start leading in to the next chapter by stating the reason we continued looking at this? or should I save that for the intro to next chapter?*


\begin{figure}
\centering
\mbox{
              \subfigure[]{\epsfig{figure=figures/GBJ1/FinalData/GF_dY.eps,width=0.5\textwidth}}\quad
              \subfigure[]{\epsfig{figure=figures/GBJ1/FinalData/GF_dY_ratio.eps,width=0.5\textwidth}}\quad
}
\caption[Gap fraction as a function of $\Delta y$ for leading $p_T$ dijet selection]{ (a) Gap fraction as a function of $\Delta y$ for various $\bar{p_T}$ slices. (b) Ratio between the theoretical predictions and data. Leading $\bar{p_T}$ dijet selection.
\label{GBJ1:dYSelA}}
\end{figure}



\begin{figure}
\centering
\mbox{
              \subfigure[]{\epsfig{figure=figures/GBJ1/FinalData/GF_ptBar.eps,width=0.5\textwidth}}\quad
              \subfigure[]{\epsfig{figure=figures/GBJ1/FinalData/GF_ptBar_ratio.eps,width=0.5\textwidth}}\quad
}
\caption[Gap fraction as a function of $\bar{p_T}$ for leading $p_T$ dijet selection]{ (a) Gap fraction as a function of $\bar{p_T}$ for various $\Delta y$ slices. (b) Ratio between the theoretical predictions and data. Leading $\bar{p_T}$ dijet selection.
\label{GBJ1:pTSelA}}
\end{figure}



\begin{figure}
\centering
\mbox{
              \subfigure[]{\epsfig{figure=figures/GBJ1/FinalData/GF_Q0.eps,width=0.5\textwidth}}\quad
              \subfigure[]{\epsfig{figure=figures/GBJ1/FinalData/GF_Q0_ratio.eps,width=0.5\textwidth}}\quad
}
\caption[Gap fraction as a function of $Q_0$ for leading $p_T$ dijet selection]{ (a) Gap fraction as a function of $Q_0$ for various $\Delta y$ and $\bar{p_T}$ slices. (b) Ratio between the theoretical predictions and data. Leading $\bar{p_T}$ dijet selection.
\label{GBJ1:Q0SelA}}
\end{figure}

\begin{figure}
\centering
\mbox{
              \subfigure[]{\epsfig{figure=figures/GBJ1/FinalData/GF_Njets_dY.eps,width=0.5\textwidth}}\quad
              \subfigure[]{\epsfig{figure=figures/GBJ1/FinalData/GF_Njets_dY_ratio.eps,width=0.5\textwidth}}\quad
}
\caption[Mean number of jets as a function of $\Delta y$ for leading $p_T$ dijet selection]{ (a) Mean number of jets as a function of $\Delta y$ for various $\bar{p_T}$ slices. (b) Ratio between the theoretical predictions and data. Leading $\bar{p_T}$ dijet selection.
\label{GBJ1:NjetsdYSelA}}
\end{figure}

\begin{figure}
\centering
\mbox{
              \subfigure[]{\epsfig{figure=figures/GBJ1/FinalData/GF_Njets.eps,width=0.5\textwidth}}\quad
              \subfigure[]{\epsfig{figure=figures/GBJ1/FinalData/GF_Njets_ratio.eps,width=0.5\textwidth}}\quad
}
\caption[Mean number of jets as a function of $\bar{p_T}$ for leading $p_T$ dijet selection]{ (a) Mean number of jets as a function of $\bar{p_T}$ for various $\Delta y$ slices. (b) Ratio between the theoretical predictions and data. Leading $\bar{p_T}$ dijet selection.
\label{GBJ1:NjetspTSelA}}
\end{figure}

%%%%%%%%%%%%%%%%%%%SELECTION B%%%%%%%%%%%%%%%%%%%%%%

\begin{figure}
\centering
\mbox{
              \subfigure[]{\epsfig{figure=figures/GBJ1/FinalData/GF_dY_SelB.eps,width=0.5\textwidth}}\quad
              \subfigure[]{\epsfig{figure=figures/GBJ1/FinalData/GF_dY_SelB_ratio.eps,width=0.5\textwidth}}\quad
}
\caption[Gap fraction as a function of $\Delta y$ for forward backward selection]{ (a) Gap fraction as a function of $\Delta y$ for various $\bar{p_T}$ slices. (b) Ratio between the theoretical predictions and data. Forward backward selection.
\label{GBJ1:dYSelB}}
\end{figure}

\begin{figure}
\centering
\mbox{
              \subfigure[]{\epsfig{figure=figures/GBJ1/FinalData/GF_SelB_pT.eps,width=0.5\textwidth}}\quad
              \subfigure[]{\epsfig{figure=figures/GBJ1/FinalData/GF_SelB_pT_ratio.eps,width=0.5\textwidth}}\quad
}
\caption[Gap fraction as a function of $\bar{p_T}$ for forward backward selection]{ (a) Gap fraction as a function of $\bar{p_T}$ for various $\Delta y$ slices. (b) Ratio between the theoretical predictions and data. Forward backward selection.
\label{GBJ1:pTSelB}}
\end{figure}

\begin{figure}
\centering
\mbox{
              \subfigure[]{\epsfig{figure=figures/GBJ1/FinalData/GF_SelB_Q0.eps,width=0.5\textwidth}}\quad
              \subfigure[]{\epsfig{figure=figures/GBJ1/FinalData/GF_SelB_Q0_ratio.eps,width=0.5\textwidth}}\quad
}
\caption[Gap fraction as a function of $Q_0$ for forward backward selection]{ (a) Gap fraction as a function of $Q_0$ for various $\Delta y$ and $\bar{p_T}$ slices. (b) Ratio between the theoretical predictions and data. Forward backward selection.
\label{GBJ1:Q0SelB}}
\end{figure}


\begin{figure}
\centering
\mbox{
              \subfigure[]{\epsfig{figure=figures/GBJ1/FinalData/Njets_SelB_dY.eps,width=0.5\textwidth}}\quad
              \subfigure[]{\epsfig{figure=figures/GBJ1/FinalData/Njets_SelB_dY_ratio.eps,width=0.5\textwidth}}\quad
}
\caption[Mean number of jets as a function of $\Delta y$ for forward backward selection ]{ (a) Mean number of jets in rapidity region as a function of $\Delta y$ for various $\bar{p_T}$ slices. (b) Ratio between the theoretical predictions and data. Forward backward selection.
\label{GBJ1:nJetsdYSelB}}
\end{figure}

\begin{figure}
\centering
\mbox{
              \subfigure[]{\epsfig{figure=figures/GBJ1/FinalData/Njets_SelB_pT.eps,width=0.5\textwidth}}\quad
              \subfigure[]{\epsfig{figure=figures/GBJ1/FinalData/Njets_SelB_pT_ratio.eps,width=0.5\textwidth}}\quad
}
\caption[Mean number of jets as a function of $\Delta y$ for forward backward selection]{ (a) Mean number of jets in rapidity region as a function of $\bar{p_T}$ for various $\Delta y$ slices. (b) Ratio between the theoretical predictions and data. Forward backward selection.
\label{GBJ1:nJetspTSelB}}
\end{figure}


\begin{figure}
\centering
\mbox{
              \subfigure[]{\epsfig{figure=figures/GBJ1/FinalData/Njets_SelB_dY_Q0pTbar.eps,width=0.5\textwidth}}\quad
              \subfigure[]{\epsfig{figure=figures/GBJ1/FinalData/Njets_SelB_dY_Q0pTbar_ratio.eps,width=0.5\textwidth}}\quad
}
\caption[Gap fraction as a function of $\Delta y$ for forward backward selection and variable $Q_0$]{ (a) Gap fraction as a function of $\Delta y$ for various $\bar{p_T}$ slices. (b) Ratio between the theoretical predictions and data. Forward backward selection with veto scale set as $\bar{p_T}$.
\label{GBJ1:pTSelB}}
\end{figure}

