
\subsection{Pileup}
\label{sec:GBJ1:Pileup}



Two effects from pileup are studied; in-time pileup and out-of-time (OOT) pileup.
In-time pileup is additional proton-proton interactions in the event and is dependent on the number of primary vertices in the event.
Out-of-time pileup is additional energy coming from previous bunch collisions and depends on both the number of primary vertices of the previous bunches and the bunch spacing. 
Both have the effect of adding additional energy into the event, which can affect the gap fraction.

The cut requiring one primary vertex will reduce the effect of in-time pileup. 
The effect of the OOT pileup and the residual in-time pileup are assessed by comparing the average gap fraction for different pileup conditions.
Table \ref{GBJ1:VertexAve} shows that the average number of primary vertices for different periods increases throughout 2010 data taking. 
The main changes in the bunch spacing are in period E where it was partially reduced and in period G where bunch trains, which had 150ns bunch spacing, were introduced.


The data periods are combined into two different bunch spacing conditions.
In periods B-D the bunch spacing was large, and so the effect of OOT pileup is expected to be small.
In periods E-I, the bunch spacing was smaller, especially from period G onwards.
It is expected that the effect of OOT pileup will be larger in this period.
The effect of OOT pileup is assessed by considering the average gap fraction for different slices in \dy{} and \ptb{} as a function of data period.
The periods B-D and E-I are separately fitted with a horizontal line, and the differences between these fits are used to assess the effect. 

To assess the residual effect from in-time pileup, the average gap fraction for different slices in \dy{} and \ptb{} as a function of data period is fitted with
\begin{equation}                              
y = A~\mathrm{and}~y = A + B x.
\label{GBJ1:Fit1}
\end{equation}
The first term has no period dependence, and the second has a period dependence.
For each function, the constants and the  $\chi^2/NDF$ are measured and assessed.
If there is no period dependence the period independent fit should do well, and the period dependent fit should have a zero gradient. 


Figures \ref{GBJ1:GFAve6070} - \ref{GBJ1:GFAve150180} show the average gap fraction as a function of period for various \dy{} and \ptb{} slices with fits to the two different OOT pileup conditions, periods B-D and E-I, with a simple $y = A$ fit. 
Figure \ref{GBJ1:GFAve6070} shows this for a low \ptb{} range of 60--70 GeV for (a) $1<\dy{}<2$, (b) $2<\dy{}<3$ and (c) $3<\dy{}<5$.
The fit for the periods EFGHI is level with the period BCD for the lowest \dy{} range, $0.2$ above for the medium \dy{} range and $0.1$ below for the \dy{} range.
For this \ptb{} range no trend is observed.
Figures \ref{GBJ1:GFAve90120} and \ref{GBJ1:GFAve150180} show a similar lack of trend and difference within $0.2$.
These results would indicate that there is no significant effect from OOT pileup. 


Table \ref{GBJ1:GFAveTable} shows the result of all the fits for the various \dy{} and \ptb{} slices. 
The $\chi^2/NDF$ of the period-independent fit show that the fits have good agreement with a straight line with no period dependence.
The period-dependent fit show the gradient of the line, $B$, is consistent, within statistical errors, with zero for almost all slices. 


These two methods have shown that after the primary vertex cut, the effect of pileup is negligible, especially when compared to the systematic uncertainties discussed later. 
 
\begin{table}
\begin{center}
\begin{tabular}{|c|c|}
\hline
Period&Ave $N^o$ Vertices\\
\hline
B&1.07\\
C&1.06\\
D&1.57\\
E&1.89\\
F&2.18\\
G&2.46\\
H&2.33\\
I&2.78\\
\hline
\end{tabular}
\caption[Average number of primary vertices for different data periods]{ 
Average number of vertices for the different data taking periods.
\label{GBJ1:VertexAve}}
\end{center}
\end{table}

\begin{figure}
\centering
\mbox{
              \subfigure[]{\epsfig{figure=figures/GBJ1/DataStability/Pileup/GFAve_060_70_1-2_exclusive.eps,width=0.5\textwidth}}\quad
              \subfigure[]{\epsfig{figure=figures/GBJ1/DataStability/Pileup/GFAve_060_70_2-3_exclusive.eps,width=0.5\textwidth}}\quad
}
\mbox{

              \subfigure[]{\epsfig{figure=figures/GBJ1/DataStability/Pileup/GFAve_060_70_3-5_exclusive.eps,width=0.5\textwidth}}\quad
                              }
\caption[Average gap fraction verses period for $60<\ptb{}<70$ GeV]{
Average gap fraction with $60<\ptb{}<70$ GeV and (a) $1<\dy{}<2$, (b) $2<\dy{}<3$  and (c) $3<\dy{}<5$. 
Each set of average gap fractions have been fitted with three horizontal lines; one for periods B-D, one for periods E-I and one for all periods.
\label{GBJ1:GFAve6070}}
\end{figure}



\begin{figure}
\centering
\mbox{
              \subfigure[]{\epsfig{figure=figures/GBJ1/DataStability/Pileup/GFAve_090_120_1-2_exclusive.eps,width=0.5\textwidth}}\quad
              \subfigure[]{\epsfig{figure=figures/GBJ1/DataStability/Pileup/GFAve_090_120_2-3_exclusive.eps,width=0.5\textwidth}}\quad
}
\mbox{

              \subfigure[]{\epsfig{figure=figures/GBJ1/DataStability/Pileup/GFAve_090_120_3-5_exclusive.eps,width=0.5\textwidth}}\quad
                              }
\caption[Average gap fraction verses period for $90<\ptb{}<120$ GeV]{
Average gap fraction with $90<\ptb{}<120$ GeV and (a) $1<\dy{}<2$, (b) $2<\dy{}<3$  and (c) $3<\dy{}<5$. 
Each set of average gap fractions have been fitted with three horizontal lines; one for periods B-D, one for periods E-I and one for all periods.
\label{GBJ1:GFAve90120}}
\end{figure}



\begin{figure}
\centering
\mbox{
              \subfigure[]{\epsfig{figure=figures/GBJ1/DataStability/Pileup/GFAve_150_180_1-2_exclusive.eps,width=0.5\textwidth}}\quad
              \subfigure[]{\epsfig{figure=figures/GBJ1/DataStability/Pileup/GFAve_150_180_2-3_exclusive.eps,width=0.5\textwidth}}\quad
}
\mbox{

              \subfigure[]{\epsfig{figure=figures/GBJ1/DataStability/Pileup/GFAve_150_180_3-5_exclusive.eps,width=0.5\textwidth}}\quad
                              }
\caption[Average gap fraction verses period for $150<\ptb{}<180$ GeV]{
Average gap fraction with $150<\ptb{}<180$ GeV and (a) $1<\dy{}<2$, (b) $2<\dy{}<3$  and (c) $3<\dy{}<5$. 
Each set of average gap fractions have been fitted with three horizontal lines; one for periods B-D, one for periods E-I and one for all periods.
\label{GBJ1:GFAve150180}}
\end{figure}



\begin{table}
\footnotesize 
\centering
\begin{tabular}{ | c | c | c | c | c | c | c | c | }
\hline
\hline
\dy{} & \ptb{} & $\bar{f}_{gap}$&\multicolumn{2}{|c|}{$y = A$} & \multicolumn{3}{|c|}{y = A + Bx}\\
   & GeV& Periods B-D &A&$\chi^{2}$/NDF&A&B&$\chi^{2}$/NDF\\
%\dy{} & $\ptb{} [GeV]$ & $\bar{f}_{gap}$ & b & c & d & e & f  \\
\hline
\input{figures/GBJ1/DataStability/Pileup/AveGFTable2.txt}
\hline
\hline
\end{tabular}
\caption[Results from period dependent and period independent fits to the average gap fraction as a function of period]{
The results from the period dependent and period independent fits (Equation \ref{GBJ1:Fit1}) for various slices of \dy{} and \ptb{}.
\label{GBJ1:GFAveTable}}
\end{table}

