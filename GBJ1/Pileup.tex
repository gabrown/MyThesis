
\subsection{Pileup}
\label{sec:GBJ1:Pileup}

The cut requiring one primary vertex will remove the effect of in-time pileup. To check the effect of out-of-time pileup, the gap fraction has been looked at as a function of data period. Each period has slightly different conditions that will effect the out-of-time pileup. It is therefore important to know both the average number of primary vertices for each period, and also any changes in the bunch spacing.  
Table \ref{GBJ1:VertexAve} shows that the average number of primary vertices for different periods increases throughout 2010 data taking. The main changes in the bunch spacing are in period E where * Need to check what happened here*, and in period G where the machine used bunch trains with 150ns bunch spacing. 
  
 
Two methods were used to assess the effect of the differing pileup conditions for the different data taking periods. Both methods are based on splitting the measurements into average gap fractions for a small range in both $\Delta y$ and $\bar{p_T}$, and seeing how this average gap fraction varies with differing pileup conditions.
The first method was just a visual check to see if the average gap fraction of the low pileup regions (periods B-D were chosen) and the high pileup regions (periods E-I) were consistent with each other and with the overall average for all periods. Figures \ref{GBJ1:GFAve6070} - \ref{GBJ1:GFAve150180} show the average gap fraction as a function of period for various $\Delta y$ and $\bar{p_T}$ slices. The fits to the different data periods was done with a simple $y = A$ fit. The figures show that any differences between the different fit lines is at the percent level, and more importantly, there is no global trend.


The second method is a more quantitative approach in which each $\Delta y$ and $\bar{p_T}$ slice is fit with 2 straight lines, one that had no dependence on period, and one that allowed a period dependent term. The two fits used are,
\begin{equation}
y = A
\label{GBJ1:Fit1}
\end{equation} 
and 
\begin{equation}
y = A + B x,
\label{GBJ1:Fit2}
\end{equation} 
 where $y$ is the average gap fraction and $x$ is the data taking period. For each fit the $\chi^2/NDF$ was calculated to check that the fit was sensible.

Table \ref{GBJ1:GFAveTable} shows the result of all the fits for the various $\Delta y$ and $\bar{p_T}$ slices. The $\chi^2/NDF$ of the fits to equation \ref{GBJ1:Fit1} show that the fits have good agreement with a straight line with no period dependence. When fitting to equation \ref{GBJ1:Fit2}, the gradient of the line, $B$, is consistent (within errors) with zero for almost all slices. 


These two methods have shown that after the primary vertex cut, the effect from pileup (mainly out-of-time) is negligible, especially when compared to the systematic uncertainties discussed later. It should be noted that this may not be the case with higher instantaneous luminosities.
 
\begin{table}
\begin{center}
\begin{tabular}{|c|c|}
\hline
Period&Ave $N^o$ Vertices\\
\hline
B&1.07\\
C&1.06\\
D&1.57\\
E&1.89\\
F&2.18\\
G&2.46\\
H&2.33\\
I&2.78\\
\hline
\end{tabular}
\caption{ Average number of vertices for the different data taking periods.}
\label{GBJ1:VertexAve}
\end{center}
\end{table}

\begin{figure}
\centering
\mbox{
              \subfigure[]{\epsfig{figure=figures/GBJ1/DataStability/Pileup/GFAve_060_70_1-2_exclusive.eps,width=0.5\textwidth}}\quad
              \subfigure[]{\epsfig{figure=figures/GBJ1/DataStability/Pileup/GFAve_060_70_2-3_exclusive.eps,width=0.5\textwidth}}\quad
}
\mbox{

              \subfigure[]{\epsfig{figure=figures/GBJ1/DataStability/Pileup/GFAve_060_70_3-5_exclusive.eps,width=0.5\textwidth}}\quad
                              }
\caption[Average gap fraction vs period for $60<\bar{p_T}<70$]{Average gap fraction with $60<\bar{p_T}<70$ and (a) $1<\Delta y<2$, (b) $2<\Delta y<3$  and (c) $3<\Delta y<5$. Each set of average gap fractions have been fitted with a horizontal line for periods B-D, periods E-I and for all periods
\label{GBJ1:GFAve6070}}
\end{figure}



\begin{figure}
\centering
\mbox{
              \subfigure[]{\epsfig{figure=figures/GBJ1/DataStability/Pileup/GFAve_090_120_1-2_exclusive.eps,width=0.5\textwidth}}\quad
              \subfigure[]{\epsfig{figure=figures/GBJ1/DataStability/Pileup/GFAve_090_120_2-3_exclusive.eps,width=0.5\textwidth}}\quad
}
\mbox{

              \subfigure[]{\epsfig{figure=figures/GBJ1/DataStability/Pileup/GFAve_090_120_3-5_exclusive.eps,width=0.5\textwidth}}\quad
                              }
\caption[Average gap fraction vs period for $90<\bar{p_T}<120$]{Average gap fraction with $90<\bar{p_T}<120$ and (a) $1<\Delta y<2$, (b) $2<\Delta y<3$  and (c) $3<\Delta y<5$. Each set of average gap fractions have been fitted with a horizontal line for periods B-D, periods E-I and for all periods
\label{GBJ1:GFAve90120}}
\end{figure}



\begin{figure}
\centering
\mbox{
              \subfigure[]{\epsfig{figure=figures/GBJ1/DataStability/Pileup/GFAve_150_180_1-2_exclusive.eps,width=0.5\textwidth}}\quad
              \subfigure[]{\epsfig{figure=figures/GBJ1/DataStability/Pileup/GFAve_150_180_2-3_exclusive.eps,width=0.5\textwidth}}\quad
}
\mbox{

              \subfigure[]{\epsfig{figure=figures/GBJ1/DataStability/Pileup/GFAve_150_180_3-5_exclusive.eps,width=0.5\textwidth}}\quad
                              }
\caption[Average gap fraction vs period for $150<\bar{p_T}<180$]{Average gap fraction with $150<\bar{p_T}<180$ and (a) $1<\Delta y<2$, (b) $2<\Delta y<3$  and (c) $3<\Delta y<5$. Each set of average gap fractions have been fitted with a horizontal line for periods B-D, periods E-I and for all periods.
\label{GBJ1:GFAve150180}}
\end{figure}



\begin{table}
\footnotesize 
\centering
\begin{tabular}{ | c | c | c | c | c | c | c | c | }
\hline
\hline
$\Delta Y$ & $\overline{p_{T}}$ Range & $\bar{f}_{gap}$&\multicolumn{2}{|c|}{$y = A$} & \multicolumn{3}{|c|}{y = A + Bx}\\
114   & GeV& Periods BCD &A&$\chi^{2}$/NDF&A&B&$\chi^{2}$/NDF\\
%$\Delta y$ & $\bar{p_T} [GeV]$ & $\bar{f}_{gap}$ & b & c & d & e & f  \\
\hline
\input{figures/GBJ1/DataStability/Pileup/AveGFTable2.txt}
\hline
\hline
\end{tabular}
\caption[Average Gap Fraction Table]{The results from the two fits (equations \ref{GBJ1:Fit1} and \ref{GBJ1:Fit2}) for various slices of $\Delta y$ and $\bar{p_T}$.
\label{GBJ1:GFAveTable}}
\end{table}

