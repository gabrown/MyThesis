
\subsection{Pileup}
\label{sec:GBJ1:Pileup}



Two effects from pileup are studied; in-time pileup and out-of-time (OOT) pileup.
In-time pileup is additional proton-proton interactions in the event and is dependant on the number of primary vertices in the event.
Out-of-time pileup is additional energy coming from previous bunch collisions and depends on both the number of primary vertices of the previous bunches and the bunch spacing. 
Both have the effect of adding additional energy in to the event, which can effect the gap fraction.

The cut requiring one primary vertex will reduce the effect of in-time pileup. 
The effect of the OOT pileup and the residual in-time pileup are assessed by comparing the average gap fraction for different pileup conditions.
Table \ref{GBJ1:VertexAve} shows that the average number of primary vertices for different periods increases throughout 2010 data taking. 
The main changes in the bunch spacing are in period E where it was partially reduced and in period G where bunch trains were introduced, which had 150ns bunch spacing. 


To check the effect of out-of-time pileup, the average gap fraction for different \dy{} and \ptb{} as a function of data period.
The data is then split into periods B, C and D which correspond to the periods with large bunch spacing, and periods E, F, G, H and I, which correspond to the periods with closer bunch spacing and bunch trains.
A horizontal line has been fitted for periods BCD and for periods EFGHI, and also for all periods.
The effect of oot pileup is then assessed by the difference in the fit lines.

To check the effect of the residual in-time pileup, the data is again split into the data period, and the average gap fraction if found for different \ptb{} and \dy{} ranges.
These average gap fraction plots as a function are fitted with two straight line functions. 
The first straight line function,
\begin{equation}
y = A,
\label{GBJ1:Fit1}
\end{equation}
 has no period dependence which corresponds to no effect from pileup.
The second straight line function,
\begin{equation}
y = A + B x,
\label{GBJ1:Fit2}
\end{equation}
has a period dependant term.
For each function, the constants and the $\chi^2/NDF$ is measured and assessed. 


Figures \ref{GBJ1:GFAve6070} - \ref{GBJ1:GFAve150180} show the average gap fraction as a function of period for various $\Delta y$ and $\bar{p_T}$ slices with fits to the different data periods with a simple $y = A$ fit. 
Figure \ref{GBJ1:GFAve6070} shows this for a low \ptb{} range of 60--70 GeV for (a) $1<\dy{}<2$, (b) $2<\dy{}<3$ and (c) $3<\dy{}<5$.
The fit for the periods EFGHI is level with the period BCD for the lowest \dy{} range, $0.2$ above for the medium \dy{} range and $0.1$ below for the \dy{} range.
For this \ptb{} range no trend is observed.
Figures \ref{GBJ1:GFAve90120} and \ref{GBJ1:GFAve150180} show a similar lack of trend and difference within $0.2$.
These results would indicate that there is no significant effect from oot pileup. 


Table \ref{GBJ1:GFAveTable} shows the result of all the fits for the various $\Delta y$ and $\bar{p_T}$ slices. 
The $\chi^2/NDF$ of the fits to equation \ref{GBJ1:Fit1} show that the fits have good agreement with a straight line with no period dependence.
When fitting to equation \ref{GBJ1:Fit2}, the gradient of the line, $B$, is consistent, within statistical errors, with zero for almost all slices. 


These two methods have shown that after the primary vertex cut, the effect of pileup is negligible, especially when compared to the systematic uncertainties discussed later. 
 
\begin{table}
\begin{center}
\begin{tabular}{|c|c|}
\hline
Period&Ave $N^o$ Vertices\\
\hline
B&1.07\\
C&1.06\\
D&1.57\\
E&1.89\\
F&2.18\\
G&2.46\\
H&2.33\\
I&2.78\\
\hline
\end{tabular}
\caption[Average number of primary vertices for different data periods]{ 
Average number of vertices for the different data taking periods.
\label{GBJ1:VertexAve}}
\end{center}
\end{table}

\begin{figure}
\centering
\mbox{
              \subfigure[]{\epsfig{figure=figures/GBJ1/DataStability/Pileup/GFAve_060_70_1-2_exclusive.eps,width=0.5\textwidth}}\quad
              \subfigure[]{\epsfig{figure=figures/GBJ1/DataStability/Pileup/GFAve_060_70_2-3_exclusive.eps,width=0.5\textwidth}}\quad
}
\mbox{

              \subfigure[]{\epsfig{figure=figures/GBJ1/DataStability/Pileup/GFAve_060_70_3-5_exclusive.eps,width=0.5\textwidth}}\quad
                              }
\caption[Average gap fraction vs period for $60<\bar{p_T}<70$]{
Average gap fraction with $60<\bar{p_T}<70$ and (a) $1<\Delta y<2$, (b) $2<\Delta y<3$  and (c) $3<\Delta y<5$. 
Each set of average gap fractions have been fitted with a horizontal line for periods B-D, periods E-I and for all periods.
\label{GBJ1:GFAve6070}}
\end{figure}



\begin{figure}
\centering
\mbox{
              \subfigure[]{\epsfig{figure=figures/GBJ1/DataStability/Pileup/GFAve_090_120_1-2_exclusive.eps,width=0.5\textwidth}}\quad
              \subfigure[]{\epsfig{figure=figures/GBJ1/DataStability/Pileup/GFAve_090_120_2-3_exclusive.eps,width=0.5\textwidth}}\quad
}
\mbox{

              \subfigure[]{\epsfig{figure=figures/GBJ1/DataStability/Pileup/GFAve_090_120_3-5_exclusive.eps,width=0.5\textwidth}}\quad
                              }
\caption[Average gap fraction vs period for $90<\bar{p_T}<120$]{
Average gap fraction with $90<\bar{p_T}<120$ and (a) $1<\Delta y<2$, (b) $2<\Delta y<3$  and (c) $3<\Delta y<5$. 
Each set of average gap fractions have been fitted with a horizontal line for periods B-D, periods E-I and for all periods.
\label{GBJ1:GFAve90120}}
\end{figure}



\begin{figure}
\centering
\mbox{
              \subfigure[]{\epsfig{figure=figures/GBJ1/DataStability/Pileup/GFAve_150_180_1-2_exclusive.eps,width=0.5\textwidth}}\quad
              \subfigure[]{\epsfig{figure=figures/GBJ1/DataStability/Pileup/GFAve_150_180_2-3_exclusive.eps,width=0.5\textwidth}}\quad
}
\mbox{

              \subfigure[]{\epsfig{figure=figures/GBJ1/DataStability/Pileup/GFAve_150_180_3-5_exclusive.eps,width=0.5\textwidth}}\quad
                              }
\caption[Average gap fraction vs period for $150<\bar{p_T}<180$]{
Average gap fraction with $150<\bar{p_T}<180$ and (a) $1<\Delta y<2$, (b) $2<\Delta y<3$  and (c) $3<\Delta y<5$. 
Each set of average gap fractions have been fitted with a horizontal line for periods B-D, periods E-I and for all periods.
\label{GBJ1:GFAve150180}}
\end{figure}



\begin{table}
\footnotesize 
\centering
\begin{tabular}{ | c | c | c | c | c | c | c | c | }
\hline
\hline
$\Delta Y$ & $\overline{p_{T}}$ Range & $\bar{f}_{gap}$&\multicolumn{2}{|c|}{$y = A$} & \multicolumn{3}{|c|}{y = A + Bx}\\
114   & GeV& Periods BCD &A&$\chi^{2}$/NDF&A&B&$\chi^{2}$/NDF\\
%$\Delta y$ & $\bar{p_T} [GeV]$ & $\bar{f}_{gap}$ & b & c & d & e & f  \\
\hline
\input{figures/GBJ1/DataStability/Pileup/AveGFTable2.txt}
\hline
\hline
\end{tabular}
\caption[Results from period dependant and period independent straight line fits to the average gap fraction as a function of period]{
The results from the two fits (equations \ref{GBJ1:Fit1} and \ref{GBJ1:Fit2}) for various slices of $\Delta y$ and $\bar{p_T}$.
\label{GBJ1:GFAveTable}}
\end{table}

