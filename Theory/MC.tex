\section{MC Event Simulation}
\label{Theory:MC}


Monte Carlo (MC) event simulation is used to compare data to theoretical predictions. 
MCs aim to give a full description of a hadron-hadron interaction.

As discussed in Section \ref{Theory:QCD}, the interaction can be factorised to separate the perturbative and non-perturbative parts. 
First the matrix element for the process is calculated to model the hard interaction.
Then high \pt{} partons from the hard interaction are evolved down to low \pt{} hadrons through parton showering and hadronization.

Parton showering (PS) evolves the partons from the ME, down to the scale, $\Lambda_{QCD}$, where pQCD no longer works and hadronization occurs.
The PS aims to describes the higher order terms concerned with soft and collinear emission of parton. 
PS are not expected to describe the wide angle hard emmision that are being probed using the jet veto. 
The emission of partons continue until all the remaining partons are at a given cut off scale.
Once at this scale, hadronization models combine partons together to make colour neutral partons.

Other emission in the MC event is underlying event (UE) and pile-up.
Underlying event (UE) is any additional radiation beyond the hard interaction within the $pp$ collision.
This mainly consists of multiple parton-parton interactions and beam remnants which add additional energy into the event.
Pile-up comes from multiple proton-proton collisions. 
Both add additional radiation into the event.




\subsection{MC Generators}
\subsubsection{PYTHIA}

PYTHIA \cite{ref:PYTHIA64} is a general-purpose MC with a large library of LO sub-processes, including the LO QCD matrix elements for the $2 \rightarrow 2$ sub-processes.
The PYTHIA PS orders emissions in \pt{} and has a veto to insure angle ordering. 
The hadronization used in PYTHIA is based on the Lund string model \cite{ref:Lund}.
The version of PYTHIA used in the analyses presented is PYTHIA 6.4.2.3 with the MRST LO$^*$ PDF \cite{ref:MRST} and the AMBT1 tune \cite{ref:Tune}

The PYTHIA sample is weighted in both leading parton \pt{} and rapidity separation between leading partons allowing to compare to data at large \ptb{} and large \dy{}.
The PYTHIA events are passed through a full ATLAS detector simulation, based on GEANT4 \cite{ref:Geant4}, allowing them to be used to unfold the detector effects and return the data to hadron scale.

%- Has library of subprocess at LO
%- for QCD has 2->2
%
%-PS: initial and final state showers
%-PS: uses decreasing timelike virtuality with angle ordering imposed by veto
%
%-UE: Descried perturbatively for set of 2 to 2 scattering
%-UE: scatterings colour connected with each other and beam reminants
%-UE: p(perp) min cut of 2GeV
%
%
%-H: Based upon the Lund String Model discussed in
%-H: Gluons is string between coloured qqbar.
%-H: As breaks, there is new q qbar pairs.

\subsubsection{HERWIG++}

HERWIG++ is another general-purpose MC which has LO QCD matrix elements for the $2 \rightarrow 2$ sub-processes.
The hadronization used in HERWIG is the cluster model \cite{ref:ClusterModel} which forces all the gluons remaining from the PS to split into quark anti-quark pairs. 
HERWIG++ has a PS which evolves the partons using angular ordering of emissions.
The version used is HERWIG++ 2.5.0 \cite{ref:HERWIG} using the MRST LO$^*$ PDF and underlying event tune of LHC-UE7-1 *reference this tune*.%\cite{ref:HERWIGTune}. 
%- Has library of subprocesses for SM and supersymetry
%- for QCD has 2->2
%
%-PS: angle ordering of parton cascade 
%
%-H: Uses 
%
%***Herwig Angle Ordering***** The first emmited gluon normally has the largest angle compared to original parton.

\subsubsection{POWHEG}
The POWHEG generator \cite{ref:Powheg1,ref:Powheg2,ref:Powheg3} has the NLO matrix elements for dijet production. 
This aims to give a better $3^{rd}$ jet description, than the standard LO MC of PYTHIA or HERWIG.
POWHEG can interface to external implementations of parton showers and  hadronization.
In this thesis, both PYTHIA and HERWIG parton showering and hadronization are used.
Events were generated with the MSTW 2008 NLO PDF.


\subsubsection{HEJ}
High Energy Jets (HEJ) \cite{ref:HEJ1,ref:HEJ2} is a generator which produces parton-level cross sections.  
HEJ implements an all-order description of wide-angle emissions, where the \pt{} of the partons are similar.
This represents the low $L$ region in Figure \ref{Theory:KineRange}.
Events were generated with the MSTW 2008 NLO PDF.
