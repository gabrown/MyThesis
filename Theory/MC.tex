\section{MC Event Simulation}
\label{Theory:MC}


Monte Carlo (MC) event simulation is used to compare data to theoretical predictions. 
MCs aim to give a full description of a hadron-hadron interaction.
As discussed in Section \ref{Theory:QCD}, the calculation can be factorised to separate the perturbative and non-perturbative parts. 
First the differential partonic cross-section is calculated to model the kinematics of the hard interaction.
This will typically consist of $n$ high \pt{} partons in the final state.
Parton showering and hadronisation algorithms are used to turn the partonic final state into a hadronic one.

PS attempts to simulate the higher order contributions to the calculation from soft and collinear parton emission.
There is a greater probability of emission for soft or collinear emission from the partons.
The effectiveness of the PS for soft and collinear emissions is tested through jet shapes \cite{ref:JetSh}.
The PS correctly resums soft and collinear contributions, ie to leading logarithms \cite{ref:Webber}.
Through the parton showering, the high \pt{} partons successively radiates partons until all partons are at a scale of a few GeV.
Once at this scale, hadronisation combines the partons together to make colour neutral hadrons. 

MC often include emission from multiple parton-parton interactions (MPI), which is included in the event before the hadronisation.



\subsection{MC Generators}
\subsubsection{PYTHIA}

PYTHIA \cite{ref:PYTHIA64} is a general-purpose MC with a large library of LO sub-processes, including the LO QCD matrix elements for the $2 \rightarrow 2$ sub-processes used for dijet production.
The PYTHIA PS orders emissions in transverse momentum and has a veto to ensure angle ordering. 
The hadronisation used in PYTHIA is based on the Lund string model \cite{ref:Lund}.
The version of PYTHIA used in the analyses presented is PYTHIA 6.4.2.3 with the MRST LO$^*$ PDF \cite{ref:MRST} and the AMBT1 tune \cite{ref:Tune}.

The dijet cross-section is falling in both \ptb{} and \dy{}.
To get a large number of events in the high \ptb{} and \dy{} regions, a filter is applied to the centrally produced PYTHIA samples used in Chapters \ref{chp:GBJ1} and \ref{chp:GBJ2}. 
The result of the filter is flat distribution in \ptb{} and \dy{}. 
Event weighting factors are stored which allows the original distribution to be recovered. 

To improve the description of data, multiple proton-proton interaction, ``pile-up'', is included.

Unlike the other MC generators considered, the PYTHIA events are passed through a full ATLAS detector simulation, based on GEANT4 \cite{ref:Geant4}, which results in a sample of fully simulated events which can be directly compared to data. 

\subsubsection{HERWIG++}

HERWIG++\cite{ref:HERWIG} is another general-purpose MC which has LO QCD matrix elements for the $2 \rightarrow 2$ sub-processes.
HERWIG++ has a PS which evolves the partons using angular ordering of emissions.
The hadronisation used in HERWIG is the cluster model \cite{ref:ClusterModel} which forces all the gluons remaining from the PS to split into quark anti-quark pairs. 
The version used is HERWIG++ 2.5.0 \cite{ref:HERWIG} using the MRST LO$^*$ PDF and underlying event tune of LHC-UE7-1 \cite{ref:Herwigpp}. 
\subsubsection{ALPGEN}

The ALPGEN \cite{ref:ALPGEN} MC generator provides leading order matrix elements for up to 6 partons in the final state.
The ALPGEN sample was generated with the CTEQ6L1 PDF set \cite{ref:ALPGENPDF}.
The sample was pass through HERWIG and JIMMY \cite{ref:Jimmy} for parton showering, hadronisation and MPI with AUET1 tune \cite{ref:ALPGENTune}.

\subsubsection{POWHEG}
The POWHEG generator \cite{ref:Powheg1,ref:Powheg2,ref:Powheg3} has the NLO matrix elements for dijet production. 
This MC is then interfaced to both PYTHIA and HERWIG for parton showering, hadronisation and MPI.
Events were generated with the MSTW 2008 NLO PDF \cite{ref:MRST}.


\subsubsection{HEJ}
High Energy Jets (HEJ) \cite{ref:HEJ1,ref:HEJ2} is a parton level generator. 
HEJ implements an all-order description of hard wide-angle emissions, in which parton emissions are ordered in rapidity.
Due to this, HEJ is expected to model the data well in the large \dy{} region in Figure \ref{Theory:KineRange}.
Events were generated with the MSTW 2008 NLO PDF.

