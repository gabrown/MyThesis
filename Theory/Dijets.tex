\section{Dijets with a Veto on Additional Radiation}

The main analysis of this thesis is dijet production with a veto on additional jet radiation between the dijets.
Dijet production cross section can be calculated using Equation \ref{Theory:XSec}.
The leading order (LO) cross section calculations have the lowest order of $\alpha_{s}$ to get the correct final state, for dijets this is $\alpha_{s}^{2}$.
Some of the LO parton scattering diagrams are shown in Figure \ref{Theory:LODijet}.
Next-to-leading order (NLO) cross section calculations are the LO cross sections with $\alpha_{s}^{2}$ and next order in $\alpha_{s}$, ie $\alpha_{s}^{3}$.
The $\alpha_{s}^{3}$ terms come from the interference between loop diagrams and LO diagrams, and also from 3 parton final state.

       \unitlength = 1mm
\begin{figure}
\centering
\mbox{
     
\begin{fmffile}{figures/QCD/gg1}
\begin{fmfgraph}(40,25)
\fmfleft{i1,i2}
\fmfright{o1,o2}
\fmf{gluon}{i1,v1,o1}
\fmf{gluon}{i2,v2,o2}
\fmf{gluon}{v1,v2}
\end{fmfgraph}
\end{fmffile}


\begin{fmffile}{figures/QCD/gg2}
\begin{fmfgraph}(40,25)
\fmfleft{i1,i2}
\fmfright{o1,o2}
\fmf{gluon}{i1,v1,o1}
\fmf{gluon}{i2,v1,o2}
\end{fmfgraph}
\end{fmffile}
     
     
    

\begin{fmffile}{figures/QCD/gg3}
\begin{fmfgraph}(40,25)
\fmfleft{i1,i2}
\fmfright{o1,o2}
\fmf{gluon}{i1,v1,i2}
\fmf{gluon}{o1,v2,o2}
\fmf{gluon}{v1,v2}
\end{fmfgraph}
\end{fmffile}

}

\enskip

\mbox{
 
   
%    Hello
%    
%\begin{fmffile}{gg4}
%\begin{fmfgraph}(40,25)
%\fmfleft{i1,i2}
%\fmfright{o1,o2}
%\fmf{phantom,tension=1.5}{i1,v1}
%\fmf{phantom,tension=1.5}{i2,v2}
%\fmf{phantom}{i1,v1,o1}
%\fmf{phantom}{i2,v2,o2}
%\fmf{gluon,tension=0}{i1,v1,o2}
%\fmf{gluon,tension=0,rubout}{i2,v2,o1}
%\fmf{gluon}{v1,v2}
%\end{fmfgraph}
%\end{fmffile}


    

\begin{fmffile}{figures/QCD/qq1}
\begin{fmfgraph}(40,25)
\fmfleft{i1,i2}
\fmfright{o1,o2}
\fmf{fermion}{i1,v1,i2}
\fmf{fermion}{o1,v2,o2}
\fmf{gluon}{v1,v2}
\end{fmfgraph}
\end{fmffile}
     
     
       

\begin{fmffile}{figures/QCD/qq2}
\begin{fmfgraph}(40,25)
\fmfleft{i1,i2}
\fmfright{o1,o2}
\fmf{fermion}{i1,v1,i2}
\fmf{gluon}{o1,v2,o2}
\fmf{gluon}{v1,v2}
\end{fmfgraph}
\end{fmffile}
     
     

\begin{fmffile}{figures/QCD/qq3}
\begin{fmfgraph}(40,25)
\fmfleft{i1,i2}
\fmfright{o1,o2}
\fmf{fermion}{i1,v1}
\fmf{fermion}{v2,i2}
\fmf{fermion,tension=0}{v1,v2}
\fmf{gluon}{v2,o2}
\fmf{gluon}{v1,o1}
\end{fmfgraph}
\end{fmffile}

}

\enskip


\mbox{



\begin{fmffile}{figures/QCD/qq4}
\begin{fmfgraph}(40,25)
\fmfleft{i1,i2}
\fmfright{o1,o2}
\fmf{fermion}{i1,v1,o1}
\fmf{fermion}{i2,v2,o2}
\fmf{gluon,tension=0}{v1,v2}
\end{fmfgraph}
\end{fmffile}
                              }
\caption[LO Feynman diagrams for dijet production]{
Some of the LO Feynman diagrams for dijet production. 
The straight lines represent quarks, and the curly lines represent gluons.
The arrows on the fermion lines going from left-to-right and right-to-left distinguish quarks and anti-quarks respectively.
\label{Theory:LODijet}}
\end{figure}





Dijet production cross section as a function of the dijet kinematic, for instance the dijet mass, and can be compared to LO and NLO cross section calculations.
The dijet cross section is compared to NLO cross section calculation in \cite{ref:Dijet}, with an agreement within the experimental uncertainty.


To study the additional radiation between the dijets, the fraction of events that do not contain an additional jet between the dijet is measured.
This is the gap fraction,
\begin{equation}
f_{gap} = \frac{\sigma_{gap}}{\sigma_{Inclusive}},
\label{Theory:GapFraction}
\end{equation}
where $\sigma_{gap}$ is the cross section for a dijet without a jet with $\pt{}>\qz{}$ in the rapidity region between them, \dr{} and $\sigma_{Inclusive}$ is the inclusive dijet cross section.
The gap fraction is studied as a function of the dijet kinematics, dijet rapidity separation \dy{} and dijet average transverse momentum \ptb{}, and the jet veto scale, \qz{}.

The dijet cross section can be calculated at fixed order using pQCD.
For some regions of the dijet kinematics, or veto jet \pt{} pQCD calculation of $\sigma_{gap}$ fails to converge.
Figure \ref{Theory:KineRange} illustrates the different effects that need to be considered for different regions of the dijet kinematics and jet veto scale.
For narrow \dr{} or small values of $L=\ln(\frac{Q}{\qz{}})$, corresponding to jet veto scale being close to the scale of the event, fixed order calculation should describe the gap cross section well.
In regions with larger separation or large values of L, resummation to all orders of perturbation theory is required.
\begin{figure}
\centering
\mbox{
              \epsfig{figure=figures/Theory/LY.pdf,width=0.9\textwidth}
}
\caption[]{
Illustration depicting what type of physics is important for the gap fraction calculation for dijet rapidity separation, Y, and $L=\ln(\frac{Q}{\qz{}})$ from \cite{ref:Forshaw_Veto}.
\label{Theory:KineRange}}
\end{figure}
%To study the emmision of radiation a veto on additional jets between the dijet can be applied.


%If an event has no jet between the dijets, it is descrived as a gap event, and 



