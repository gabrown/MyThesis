\section{QCD}
\label{Theory:QCD}

The expression for the QCD Lagrangian which describes interactions between spin-1/2 quarks, $q$, with mass $m$ interacting with spin-1 gluons is 
\begin{equation}
\mathcal{L} = - \frac{1}{4} F^{a}_{\alpha\beta} F_{b}^{\alpha\beta} + \sum_{q} \bar{q}_{j}( i{\not}\partial - m_{q})q^{j}  + \sum_{q} \bar{q}_{i}\gamma_{\mu} t^{a}_{ik} q^{k} A^{\mu}_{a}, 
\label{Theory:L_QCD}
\end{equation}
where $A^{\mu}_{a}$ is the gluon field, $t^{a}_{ik}=\frac{1}{2}\lambda_{a}$ where $\lambda_{a}$ are the Gell--Mann matrices, $\gamma_{\mu}$ are the gamma matrices and 
\begin{equation}
F^{a}_{\alpha\beta} = \partial^{\alpha}A_{A}^{\beta} - \partial^{\beta}A_{a}^{\alpha} + g f_{abc} A_{b}^{\alpha}A_{c}^{\beta}. 
\label{Theory:F_Tensor}
\end{equation}
Repeated indices implies summation over them, with greek charachters representing lorrentz vertor components and latin carachters representing the different colour states for quarks and gluons. 

The first term in Equation \ref{Theory:L_QCD} is concerned with the gluon self coupling and gluon propagator, where terms with $g^2$  correspond to a four gluon interaction, terms with $g$ correspond to the three gluon interaction and other terms correspond to the basic gluon propagator.
In Equation \ref{Theory:F_Tensor}, the third term is a key difference to QED, and allows for the self interactions of the gluon.
The second term in Equation \ref{Theory:L_QCD} corresponds to the basic quark propagator without a gluon interaction.
The final term in Equation \ref{Theory:L_QCD} is the gluon-quark interaction.

\subsection{Asymptotic Freedom}


The coupling constant, $\alpha_{s} = \frac{g_s^2}{4\pi}$, governs the strength of the partonic interactions and is dependent on the scale, Q, of the interaction.
When interactions are calculated using perturbative QCD (pQCD), ultraviolet divergences arise which require a renormalisation. 
This procedure introduces an additional mass scale, $\mu$.
The coupling constant at a momentum scale, Q, relative to a scale $\mu$ at one-loop order is given by
\begin{equation}
\alpha_{s}(Q^2) = \frac{\alpha_{s}(\mu^2)}{1+b\alpha_{s}(\mu^2)\ln(\frac{Q^2}{\mu^2})}
\label{Theory:Coupling}
\end{equation}
where 
\begin{equation}
b = \frac{33-2n_f}{12\pi}
\label{Theory:b_Coupling}
\end{equation}
where $n_f$ is the number of active flavours of quarks.
$b$ is positive for all flavours of quark in the SM.

The renormalisation scale $\mu$ is arbitrary and so can be freely chosen.
Measurements of $\alpha_s$ are made at various values of $Q$ and are compared at $\alpha_s(M_Z)$, where $M_Z$ is the mass of the Z.
This allows the calculation of $\alpha_s$ at other scales, though as Q gets small, Equation \ref{Theory:Coupling} diverges and pQCD is no longer appropriate.

From Equation \ref{Theory:Coupling}, as the value of $Q$ increases, corresponding to probing smaller distances, the coupling constant becomes smalland the quarks and gluons behaves as if they were free particles.
QCD has asymptotic freedom.

Another important feature of QCD is confinement, which is the requirement that quarks and gluons are always found in bound colour singlet states, namely hadrons.
Equation \ref{Theory:Coupling} also hints towards confinement when $Q$ get small the coupling becomes large, though at this point the higher order terms need to be taken into account in Equation \ref{Theory:Coupling}. 


pQCD should only be used down to a scale $\Lambda_{QCD}$ at which higher orders of $\alpha_s$ become significant and the perturbative expansion no longer converge.
When modelling events, the different scales can be used to define a perturbative region, where the calculation can be done using pQCD, and a non-perturbative region, where perturbation theory is no longer reliable.
%This separation, or factorisation, of the different scales will be discussed further in Sections XXX and XXX.




\subsection{Hadron -- Hadron Cross Section}


As discussed in \cite{ref:HardInt}, QCD factorisation theorem indicates that the hadronic cross section of a given process can be split into the hard partonic scattering process, which can be calculated using pQCD, and a part that describes the non-perturbative structure of the hadron, characterised using the parton distribution functions (PDFs).


The cross section for the process shown in Figure \ref{Theory:HadronHadron} is given by,

\begin{equation}
\sigma_{AB}=\int\int dx_a dx_b f_{a/A}(x_a) f_{b/B}(x_a) \hat{\sigma}_{ab\rightarrow X}
\label{Theory:XSec}
\end{equation}
where $A$ and $B$ are the initial protons, $a$ and $b$ are different combinations of quarks and gluons, $x_a$ is the fraction of the proton energy that parton $a$ carries, $ f_{a/A}(x_a)$ is the PDF which, to leading order, represents the probability of finding a parton with energy fraction  $x_a$ within A, and $\hat{\sigma}_{ab\rightarrow X}$ is the partonic cross section for the sub-process $ab\rightarrow X$.
The partonic scatter is described by the matrix element, $\mathcal{M}$, and the resulting partonic cross section is given by
\begin{equation}
d \hat{\sigma}_{a,b\rightarrow X}= \frac{1}{\hat{s}}|\mathcal{M}_{a,b\rightarrow X}|^{2} d\Phi_n
\label{Theory:PartonCrossSection}
\end{equation}
where $a$ and $b$ represent the incoming partons, $\hat{s}= (k_a + k_b)^2$ where $k_i$ is the momentum of a parton $i$, and $d\Phi_n$ is the n-body phase space.
The partonic cross section is calculated using the Feynman rules.
When $X$ is just quarks and gluons, then the Feynman rules can be derived only from the QCD Lagrangian. 

Problems occur when calculating collinear or soft gluon emission from $a$ or $b$, because infra-red (IR) divergences occur.  
A factorisation scale $\mu_F$ is defined such that emissions with a momentum less than $\mu_F$ are absorbed into the PDFs, and only emissions above this value are calculated. 

%Once the divergences are absorbed into the PDFs and the $\hat{\sigma}_{ab\rightarrow X}$ is perturbative expanded, Equation \ref{Theory:XSec} becomes,
%\begin{equation}
%\sigma_{AB}=\int\int dx_a dx_b f_{a/A}(x_a,\mu_F) f_{b/B}(x_b,\mu_F) [\hat{\sigma}_0 + \alpha_s(\mu_R^2)  * \hat{\sigma}_1 + HO]_{ab\rightarrow X}
%\label{Theory:XSec2}
%\end{equation}
%where $\hat{\sigma}_0$ and $\hat{\sigma}_1$ are the LO and NLO cross section respectively, and  $\alpha_s(\mu_R)$ is the renormalisation scale.
%Both the factorisation and renormalisation scale is set to be close to scale of the hard interaction of the event.
%This choise for the $\mu_F$ and $\mu_R$ reduces the effect from higher order terms.


The PDFs, $f(x_0,Q_0)$, have been measured at electron-proton collider and fix target experiments for a range of $x_0$ and $Q_0$ values. 

\begin{figure}
\centering
\mbox{
              \epsfig{figure=figures/Theory/Hadron-Hadron.eps,width=0.6\textwidth}
                              }
\caption[]{
Illustration showing hadron hadron interaction through partons $a$ and $b$ going to $X$. 
\label{Theory:HadronHadron}}
\end{figure}


\subsection{Jet Formation}

For a parton scatter with partonic final state objects, the coloured partons move away from the interaction point.
As discussed above, confinement requires that all observable particles are colour neutral, and this will only occur when there is a low relative momentum between partons.
As the distance between the partons increases, pair production and gluon emission create a cascade of partons. 
This results in final state hadrons that can be detected.

A jet is defined to allow calculations with partons in the final state to be compared to calculations that have a hadronic final state, and also to detector level objects.
The jet definition is required to give a duality between the parton level and hadron level. 
Figure \ref{Theory:InitJet} is an illustration of a jet at different levels.


\begin{figure}
\centering
\mbox{
              \epsfig{figure=figures/Theory/JetEvolution.eps,width=0.9\textwidth}
                              }
\caption[]{
Illustration of a jet at parton, particle and calorimeter levels. 
Illistration from Dag Gillberg.
\label{Theory:InitJet}}
\end{figure}

