       \unitlength = 1mm
\begin{figure}
\centering
\mbox{
     
\begin{fmffile}{figures/QCD/gg1}
\begin{fmfgraph}(40,25)
\fmfleft{i1,i2}
\fmfright{o1,o2}
\fmf{gluon}{i1,v1,o1}
\fmf{gluon}{i2,v2,o2}
\fmf{gluon}{v1,v2}
\end{fmfgraph}
\end{fmffile}


\begin{fmffile}{figures/QCD/gg2}
\begin{fmfgraph}(40,25)
\fmfleft{i1,i2}
\fmfright{o1,o2}
\fmf{gluon}{i1,v1,o1}
\fmf{gluon}{i2,v1,o2}
\end{fmfgraph}
\end{fmffile}
     
     
    

\begin{fmffile}{figures/QCD/gg3}
\begin{fmfgraph}(40,25)
\fmfleft{i1,i2}
\fmfright{o1,o2}
\fmf{gluon}{i1,v1,i2}
\fmf{gluon}{o1,v2,o2}
\fmf{gluon}{v1,v2}
\end{fmfgraph}
\end{fmffile}

}

\enskip

\mbox{
 
   
%    Hello
%    
%\begin{fmffile}{gg4}
%\begin{fmfgraph}(40,25)
%\fmfleft{i1,i2}
%\fmfright{o1,o2}
%\fmf{phantom,tension=1.5}{i1,v1}
%\fmf{phantom,tension=1.5}{i2,v2}
%\fmf{phantom}{i1,v1,o1}
%\fmf{phantom}{i2,v2,o2}
%\fmf{gluon,tension=0}{i1,v1,o2}
%\fmf{gluon,tension=0,rubout}{i2,v2,o1}
%\fmf{gluon}{v1,v2}
%\end{fmfgraph}
%\end{fmffile}


    

\begin{fmffile}{figures/QCD/qq1}
\begin{fmfgraph}(40,25)
\fmfleft{i1,i2}
\fmfright{o1,o2}
\fmf{fermion}{i1,v1,i2}
\fmf{fermion}{o1,v2,o2}
\fmf{gluon}{v1,v2}
\end{fmfgraph}
\end{fmffile}
     
     
       

\begin{fmffile}{figures/QCD/qq2}
\begin{fmfgraph}(40,25)
\fmfleft{i1,i2}
\fmfright{o1,o2}
\fmf{fermion}{i1,v1,i2}
\fmf{gluon}{o1,v2,o2}
\fmf{gluon}{v1,v2}
\end{fmfgraph}
\end{fmffile}
     
     

\begin{fmffile}{figures/QCD/qq3}
\begin{fmfgraph}(40,25)
\fmfleft{i1,i2}
\fmfright{o1,o2}
\fmf{fermion}{i1,v1}
\fmf{fermion}{v2,i2}
\fmf{fermion,tension=0}{v1,v2}
\fmf{gluon}{v2,o2}
\fmf{gluon}{v1,o1}
\end{fmfgraph}
\end{fmffile}

}

\enskip


\mbox{



\begin{fmffile}{figures/QCD/qq4}
\begin{fmfgraph}(40,25)
\fmfleft{i1,i2}
\fmfright{o1,o2}
\fmf{fermion}{i1,v1,o1}
\fmf{fermion}{i2,v2,o2}
\fmf{gluon,tension=0}{v1,v2}
\end{fmfgraph}
\end{fmffile}
                              }
\caption[LO Feynman diagrams for dijet production]{
Some of the LO Feynman diagrams for dijet production. 
The straight lines represent quarks, and the curly lines represent gluons.
The arrows on the fermion lines going from left-to-right and right-to-left distinguish quarks and anti-quarks respectively.
\label{Theory:LODijet}}
\end{figure}




