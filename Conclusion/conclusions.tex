In this thesis a summary of two precision jet measurements are presented that aim to probe higher-order QCD phenomena by studying the amount of radiation from a dijet system.

To make precision jet measurements, a good understanding of the detector and its response to jets is vital. 
The monitoring of the calorimeter high-level trigger, which is used in the analyses to trigger events, has been detailed. 
The trigger was shown to be stable over different data periods, and problems, such as noisy cells, could be monitored and masked.
The performance of jets within the calorimeter was also assessed.
The dijet \pt{} balance method was used to extend a determination of the central jet energy scale uncertainty out to larger \dy{}, and the effect from pileup and a closure of the method were studied. 
Additionally, some properties of jets in the transition region between the end-cap and FCal and in the FCal were studied and compared to different showering models.
The closure of the dijet \pt{} balance method and the properties of the jets in the transition region between the end-cap and FCal and in the FCal itself have been  published as an ATLAS conference note \cite{ref:EtaInter2010}.

Dijet production with a jet veto for fixed regions of phase space was studied for rapidity separations of up to six units in rapidity with average \pt{} of the dijets from 50 to 500 GeV.  
The measurement of the fraction of events that survived a jet veto were compared to PYTHIA, HERWIG++ and ALPGEN event generators as well as to next-to-leading-order predictions from POWHEG, interfaced with the partons showering from PYTHIA and HERWIG, and to a prediction using the HEJ generator.
Two different dijet selections were studied, the leading \pt{} dijet selection and the forward/backward dijet selection. 
No prediction agreed in all the areas of phase-space considered.
POWHEG + PYTHIA had the best agreement with data, but differences were observed at high \dy{}.
HEJ described the data as a function of \dy{}, but only for low $\ptb{}$; at high $\ptb{}$ the gap fraction was too high.
POWHEG + HERWIG gave a poor description of the data with too low a gap fraction throughout.
The mean multiplicity of jets in the rapidity region has also been presented for the leading \pt{} dijet selection.
The activity of HEJ was significantly lower than the data for all but the lowest \dy{} bin. 
Given the agreement for the gap fraction, it seems HEJ describes the veto jet well, but does not cope well with any additional jets beyond that. 
POWHEG + HERWIG had too much activity throughout, which correlates well with the predicted gap fraction that is too low.
POWHEG + PYTHIA gave the best description of the data, although they did deviate at high \dy{}.
The results of this analysis have been published in JHEP \cite{ref:ATLASGap} as well in the ATLAS conference notes \cite{ref:GBJConf1,ref:GBJConf2}.

A preliminary analysis studying  emissions from very high rapidity separated jets was also presented, with a measurement up to a separation of eight units in rapidity.
The systematic uncertainties from the data selection and jet uncertainties were assessed.
In the analysis both the gap fraction and the mean number of jets in the rapidity region between the dijet system were studied. 
The azimuthal decorrelation variables, \dphiDist{}, \mean{\cosdphi} and \mean{\costwodphi} were also studied.
The data were compared to fully reconstructed PYTHIA.
In the high \dy{} region, the gap fraction levelled off in the data. 
This feature was only observed at high \pt{} in the previous analysis. 
The shape of the PYTHIA distributions in the variables \mean{\cosdphi} and \mean{\costwodphi} were different to the data, with the data having a lower value at very high \dy{}.  
The results of this second analysis are currently being reviewed internally by the ATLAS collaboration \cite{ref:GBJInternal}.


