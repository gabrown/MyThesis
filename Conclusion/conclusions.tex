In this thesis a summary of two precision jet measurements are presented that aim to probe higher order QCD phenomena by studying the amount of radiation from a dijet system.

To make precision jet measurements, a good understanding of the detector and its response to jets is vital. 
Presented is the monitoring of the calorimeter high level trigger which is used in the analyses to trigger events. 
It is shown to be stable of data periods, and problems such as noisy cells can be observed and masked.
The performance of jets within the calorimeter is also assessed.
The dijet \pt{} balance method is used to extend the central uncertainty out to larger \dy{}, and the effect from pileup and a closure of the method is studied. 
Additionally, some jet properties for jets in the transition region between the end-cap and FCal and in the FCal are studied and compared to different showering models.

A measurement of dijet production with a jet veto for fixed areas in phase space is first studied for rapidity separations of up to six units in rapidity and of average \pt{} of the dijets from 50 -- 500 GeV.  
The measurement of the fraction of events that survive a jet veto are compared to PYTHIA, HERWIG++ and ALPGEN event generators as well as next-to-leading order prediction from POWHEG, interfaced with the partons showering from PYTHIA and HERWIG, as well as a prediction using the HEJ generator.
Two different dijet selections were studied, the leading \pt{} dijet selection and the forward/backward dijet selection. 
No prediction agreed in all the areas of phase-space considered.
POWHEG + PYTHIA agreed the best with data, but differences were observed at high \dy{}.
HEJ described the data as a function of \dy{}, but only for low $\ptb{}$, at high $\ptb{}$ the gap fraction was too high.
POWHEG + HERWIG gave a poor description of the data with to low a gap fraction throughout.
The mean multiplicity of jets in the rapidity region is also presented for the leading \pt{} dijet selection.
The activity of HEJ was significantly lower than the data for all but the lowest \dy{} bin. 
Given the agreement for the gap fraction, it seems HEJ describes the veto jet well, but not any additional jets beyond that. 
POWHEG + HERWIG had too much activity throughout, which correlates well with the gap fraction.
POWHEG + PYTHIA gave the best description of the data, although did deviate at high \dy{}.

In the second analysis presented, the aim was to study emissions from very high rapidity separated jets, with measurements up to eight units in rapidity separation.
HEJ was interfaced with the \pt{} ordered parton shower Ariadne.
Again the gap fraction and multiplicity of jets in the rapidity region were studied, but also azimuthal decorralation observables, \mean{\cosdphi} and \mean{\costwodphi}.
The gap fraction as a function of \dy{} levelled off at large \dy{}. 
This may have been due to PDF effects, but another explanation could be colour singlet exchange, this will need to be studied in detail.
POWHEG + PYTHIA described the data best, although the gap fraction was below the data.
HEJ which in the previous analysis described the gap fraction against \dy{} well, no longer agrees with the data. 
This could be due the change in the dijet kinematic cuts, the addition of the parton shower, or missing physics, such as colour singlet exchange.
The difference in HEJ continue when the multiplicity of jets was studied.
Where before HEJ had too low an activity, with the addition of the parton shower, the activity was too high.
\mean{\cosdphi} and \mean{\costwodphi} were also studied as a function of gap and inclusive events.
These observables did not work as a good discriminator of the different predictions, with both POWHEG curves and HEJ curve falling close to the data.
HEJ did have a different shape of distribution for the inclusive events.


