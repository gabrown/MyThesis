The internal jet structure of jets in the forward region was examined using 2010 data to assess how well the MC simulation reproduces the data.
The transverse size of the jet is quantified using the jet width,
\begin{equation}
\rm{Width}=\frac{\sum (r^{cluster} \times \et{}^{cluster}) }{\sum \et{}^{cluster}},
\label{JetPerf:Width}
\end{equation}
where the sums are over all clusters within the jet, and $r^{cluster}$ is the distance of each cluster from the jet centre.
Another important jet observable is the electromagnetic fraction, which is the fraction of the total jet energy (at EM scale) coming from electromagnetic clusters.


Figure \ref{JetPerf:Width_EMF} (a) shows the jet width for jets in the forward region.
Figure \ref{JetPerf:Width_EMF} (b) shows the EMF for jets in the forward region.
In both plots the data are compared to PYTHIA using three different physics lists which use different calorimeter interaction models.
The three physics lists used, QGSP, $\rm QGSP\_BERT$, and $\rm FTFP\_BERT$ are discussed in \cite{ref:HadModels}. 
These lists define different aspects of modelling the interactions of hadrons with matter, which are shown in Table \ref{JetPerf:Models}.

The QGSP physics list contains the quark gluon string model, which is a phenomenological model describing the parton production arising from collisions between hadrons and nucleon, for high energy hadrons, and uses a low energy parameterisation model (LEP), which is a parametrised model that is based on extrapolating measured reaction cross-sections for the low energy hadrons.  
The $\rm{QGSP\_BERT}$ physics list still uses QGSP at high hadronic energy, but only uses LEP at medium energies.
For low energy hadrons, the Bertini nucleon-nucleon scattering model (BERT) is used, which is an alternative model for the low energy interaction of hadrons in  the nucleon medium.
$\rm{FTFP\_BERT}$ uses the  Fritiof fragmentation model (FTF) to model the high energy interactions. 
In regions where there is overlap between the different models, there is linear interpolation between them.

%Table \ref{JetPerf:Models} takes the pertinent parts from Table 6 in \cite{ref:HadModels}.

%The important aspects of the modelling of interaction of particles with the detector for this comparison are nucleus fragmentation and intra-nuclear cascades.
%Intra-nuclear cascades are relevant for energies of $1-10$ GeV, and nucleus fragmentation are important for energies above that.
%
%Nucleus fragmentation describes the parton production arising from collisions between hadrons and nucleons in the detector. 
%Two different ways of modelling the fragmentation are parametrised modelling, which attempts extrapolate measured reaction cross-sections, and phenomenological models based around strings. 
%Low energy parametrised model, LEP, is an example of parametrised modelling. 
%The two phenomenological models considered are quark gluon string model (QGS), and Fritiof fragmentation model (FTF).
%
%Intra-nuclear cascades models describe the interaction of hadrons in the nucleon medium.
%Only the Bertini nucleon-nucleon scattering model (BERT) is considered.

By comparing the standard PYTHIA physics list, $\rm QGSP\_BERT$, to QGSP and $\rm FTFP\_BERT$, the effects of removing the BERT model and also changing from the QGS model to the FTF model can be seen.

\begin{table}
\centering
\begin{tabular}{ | c | c | c | c |}
Physics List& \multicolumn{3}{ c |}{Hadron Energy Range (GeV)} \\ 
& Low & Medium & High \\ 
\hline
           QGSP    &                       &    $0-25$ LEP     &    $>12$ QGSP \\
$\rm QGSP\_BERT$   &    $0-9.9$ GeV BERT   &    $9.5-25$ LEP   &    $>12$ QGSP \\
$\rm FTFP\_BERT$   &    $0-5$ GeV BERT     &                   &    $>4$ FTF   \\
\end{tabular}
\caption[Physics lists description of hadron interaction models used for various hadron energies]{
Hadron interaction models for different physics list for various hadron energies.
Taken from Table 6 in \cite{ref:HadModels}.
\label{JetPerf:Models}}
\end{table}

None of the physics list manage to accurately describe the data, and the width is consistently higher in data than the simulations. 
These differences have been observed for central jets also, though with smaller differences \cite{ref:JetShapes}.
As concluded in \cite{ref:HadModels}, the physics lists chosen to be default at ATLAS produced narrower and shorter showers than data, but gave the best agreement with the data for the pion response.
These results were published in an ATLAS conference note \cite{ref:EtaInter2010}. 

\begin{figure}
\centering
\mbox{
              \subfigure[]{\epsfig{figure=figures/JetPerformance/Width.eps,width=0.5\textwidth}}\quad
              \subfigure[]{\epsfig{figure=figures/JetPerformance/EMF.eps,width=0.5\textwidth}}\quad
}
\caption[]{(a) Jet width and (b) jet EMF for jets with \pt{}$>20$ GeV and in the region \etaRange{3.2}{4.5}.
2010 data is compared to standard PYTHIA with the different physics lists, $\rm QGSP\textunderscore{}BERT$ (yellow filled), $\rm FTFP\textunderscore{}BERT$ (red circles) and $\rm QGSP$ (purple circles).  
\label{JetPerf:Width_EMF}}
\end{figure}

