In 2010, the dijet \pt{} balence was not used to recalibrate the jets, but was used to assess the JES calibration factors.
The relative jet response, \cite{ref:EtaInter2010}, shows a different in the forward region for MC and data.

The difference between the data and MC jet responses could either originate from physics or detector effects. 
If the difference if from a detector effect, for instance .., then the calibration should be applied to the data to correct for it.
If the difference is actually from a physics effect in the MC, for instance ..., then it should be added as an uncertainty in the method.



Using the ratio of the relative response factors from the MC and the data, the residual correction, 

\begin{equation}
\mathcal{C_{\rm{in-situ}}} = \frac{\rm{c_{MC}}}{\rm{c_{Data}}}
\label{JetPerf:ResidualCorrection}
\end{equation}
is calculated, where $\frac{1}{\rm{c_{MC}}}$ and $\frac{1}{\rm{c_{Data}}}$ are the relative response factors for MC and data respectively. 


Figure \ref{JetPerf:EtaData} shows the jet $\eta$ distribution for (a) \pt{}$>30$ GeV from the Min Bias trigger and (b) \pt{}$>50$ from the calorimeter trigger stream. 
The in-situ calibrated data has a much better agreement with the MC than the regular data does in both plots. 
There are still differences in the forward region, but overall the differences are smaller than ten percent.

Figure \ref{JetPerf:E_PtData} is the jet (a) \pt{}  and  (b) energy distributions  fot jets in the region \etaRange{3.2}{4.5}  for \pt{}$>20$ GeV. 
The energy distribution shows improvements in agreement between the MC after the data is calibrated.
The \pt{} distribution shows smaller differences for the in-situ calibrated data.



\begin{figure}
\centering
\mbox{
              \subfigure[]{\epsfig{figure=figures/JetPerformance/EtaDist30GeV.eps,width=0.5\textwidth}}\quad
              \subfigure[]{\epsfig{figure=figures/JetPerformance/EtaDist50GeV.eps,width=0.5\textwidth}}\quad
}
\caption[]{
$\eta$ distribution for jets with (a) \pt{}$>30$ GeV from the Min Bias trigger and (b) \pt{}$>50$ from the calorimeter trigger stream. 
Uncorrected data (open black circles) and corrected data (red circles) are shown along with the reference MC. 
\label{JetPerf:EtaData}}
\end{figure}

\begin{figure}
\centering
\mbox{
              \subfigure[]{\epsfig{figure=figures/JetPerformance/EDist.eps,width=0.5\textwidth}}\quad
              \subfigure[]{\epsfig{figure=figures/JetPerformance/PtDist.eps,width=0.5\textwidth}}\quad
}
\caption[]{
(a) Jet energy and (b) jet \pt{} for the region \etaRange{3.2}{4.5}.
Uncorrected data (open black circles) and corrected data (red circles) are shown along with the reference MC. 
\label{JetPerf:E_PtData}}
\end{figure}




