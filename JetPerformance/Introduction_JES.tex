The response of the calorimeter to jets is found using fully simulated MC, comparing reconstructed jets to truth jets.
These correction factors are known as the JES calibration and are described in Section \ref{sec:Det:Jets}. 
To obtain the uncertainty on the JES, various MC and in-situ validations are done.
This chapter describes one such in-situ method, \pt{} balancing of dijet events. 

In the central region (\etaRange{0}{0.8}), the JES and its uncertainty  are well known due to test-beam data and good knowledge of the detector geometry.
The \pt{} balancing of dijet events can extend the determination of the uncertainty into the end-cap (\etaRange{0.8}{2.8}) and forward (\etaRange{2.8}{4.5}) regions. 
The jets in the central region, which are well understood, are used to obtain relative JES uncertainties in the end-cap and forward regions. 

This method of extrapolating the uncertainty outside the central region is achieved by using the \pt{} balance of dijet events to derive a relative jet response between the two jets.
In a dijet topology, it is expected that both jets should have the same transverse momentum, assuming that the jets arise from a $2\rightarrow2$ partonic scatter.
Using this assumption, the \pt{} imbalance can be expressed using the asymmetry $\mathcal{A}$ defined as
\begin{equation}
\mathcal{A} =\frac{\pt{}^1 - \pt{}^2}{\ptave{}},
\label{JetPerf:Assym}
\end{equation}
where $\pt{}^1$ and $\pt{}^2$ are the transverse momentum of the two leading jets, and $\ptave{}$ is the average \pt{} of the two jets.
For imperfectly measured jets, the asymmetry would not be unity and a relative calorimeter response to a jet can be constructed using the asymmetry as,
\begin{equation}
 \frac{\pt{}^1}{\pt{}^2}=\frac{2+\mean{\mathcal{A}}}{2-\mean{\mathcal{A}}}.
\label{JetPerf:JetResp}
\end{equation}

The assumption made is that the event comes from a $2\rightarrow2$ partonic scatter, without an additional hard QCD emission in the interaction.
This assumption can be wrong if there are more than two hard jets, or if there is a large amount of soft quark or gluon emission outside the jets. 
To reduce these effects, a cut is placed on the third jet \pt{} of $\pt{}_3<0.25\times\ptave{}$ and also requiring the jets to be back-to-back in azimuth with a cut $\dphi{}>2.6$.
 
