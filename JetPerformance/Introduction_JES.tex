The response of the calorimter to jets can be defined using fully simulated MC samples as
\begin{equation}
\mathcal{R(\eta)} = \frac{\pt{}^{reco}(\eta)}{\pt{}^{truth}(\eta)}
\label{JetPerf:MCJetResp}
\end{equation}
where \pt{}$^{reco}$ is the \pt{} of the MC jet after full simulation of the detector, and  \pt{}$^{truth}$ is the \pt{} of the MC jet, before detector simulation. 
The truth and reco jets are matched using a \dr{} cut of 0.3.
%The calibration is done using the H1 method, which is described in \cite{ref:H1}. 

Jet response within the calorimeter varies as a function of $\eta$ due to the different calorimeter technology used and the varying levels of dead material, as shown in Figure \ref{JetPerf:MCResp}.
The default calibration procedure in ATLAS is to correct the energy and momentum by $\frac{1}{\mathcal{R(\eta)}}$.

\begin{figure}
\centering
\mbox{
              {\epsfig{figure=figures/JetPerformance/MCResponse-ref_ATLAS-CONF-2010-055.eps,width=\textwidth}}
                              }
\caption[]{ PYTHIA simulated jet EM-scale response as a function of reconstructed jet $\eta$ for different jet energies \cite{ref:EtaInter2010}
\label{JetPerf:MCResp}}
\end{figure}

The accuracy of the calibration depends how well the MC models the physics, such as ..., and how accurately if describes the detector geometry. 
In-situ methods are used to check the calibration and to assign a corresponding uncertainty.


********Add a section on the JES uncertainty contributions and link to paper ******
In the central region (\etaRange{0}{0.8}), the response and uncertainty from single particles in the calorimeter is used to give the JES uncertainty.  
This is done by  studying the charged particle response of the detector, gained from comparing charged tracks to isolated energy deposits.
********


Outside of this central region different in-situ methods are used to get contributions to the JES uncertainty.
One of the in-situ methods used to assess the uncertainty in the end cap (\etaRange{0.8}{2.8}) and forward (\etaRange{2.8}{4.5}) regions is pseudorapidity inter-calibration using dijets.
The jets in the central region, which are well understood, are used as a baseline to get relative calibration uncertainties in the end cap and forward regions. 

This method of extending the uncertainty outside the central region is achieved by using the \pt{} balance of dijet events to get a relative jet response between the 2 jets.
In a dijet topology, it is expected, that both jets should have the same transvere momentum, assuming that the jets arise from a $2\rightarrow2$ partonic scatter.
Using this assumption, the \pt{} imbalance can be expressed using the asymmetry, $\mathcal{A}$, defined as
\begin{equation}
\mathcal{A} =\frac{\pt{}^1 - \pt{}^2}{\ptave{}},
\label{JetPerf:Assym}
\end{equation}
where $\pt{}^1$ and $\pt{}^2$ are the transverse momentum of the two leading jets, and $\ptave{}$ is the average \pt{} of the two jets.
For imperfectly measured jets, the asymmetry would not be unity and a relative calorimeter response to a jet can be constructed using the asymmetry as,
\begin{equation}
 \frac{\pt{}^1}{\pt{}^2}=\frac{2+\mean{\mathcal{A}}}{2-\mean{\mathcal{A}}}.
\label{JetPerf:JetResp}
\end{equation}


For the balancing to work it is important that there is no other hard QCD emission in the interaction, this is achieved with a cut on the 3rd jet $\pt{}^3<0.25*\ptave{}$ and requiring the jets to be back-to-back in azimuth with a cut $\dphi{}>2.6$.

 
