\subsubsection{Matrix Method for Dijet Balance}

The matrix method differs from the standard method by not requiring a specific reference region.
Instead of a probe and reference jet, there is a ``left'' jet and a ``right'' jet, where $\eta_{\rm left}<\eta_{\rm right}$.
As a result Equations \ref{JetPerf:Assym} and \ref{JetPerf:JetResp} become,
\begin{equation}
\mathcal{A} =\frac{\pt{}^{left} - \pt{}^{right}}{\ptave{}}
\label{JetPerf:MM_Assym}
\end{equation}
and 
\begin{equation}
\frac{\pt{}^{left}}{\pt{}^{right}}=\frac{2+\mathcal{A}}{2-\mathcal{A}}
\label{JetPerf:MM_CorrectionFactor}
\end{equation}

For given values of $\eta^{left}$ and $\eta^{right}$, the relative response between the two regions can be defined as 

\begin{equation}
\mathcal{R_{\rm ijk}}= \frac{2-\mean{\mathcal{A_{\rm ijk}}}}{2+\mean{\mathcal{A_{\rm ijk}}}} = \frac{ c_{ik}^{left}}{ c_{jk}^{right}}
\label{JetPerf:MM_Response_Measured}
\end{equation}
where i, j and k are label bins in  $\eta^{left}$, $\eta^{right}$ and $\ptave{}$ respectively, and $\mean{\mathcal{A}}$ is the mean \footnote{The asymmetry distribution is fitted with a Gaussian function between -0.7 and 0.7, and the value of the fit is taken as the mean, unless there are low statistics, then the average of the asymmetry is taken} of the asymmetry distribution. 


For every k-th $\ptave{}$ bin, there exists $\sum\limits^N_{n=1}n$ relative responses, $\mathcal{R_{\rm ijk}}$, corresponding to different $\eta^{left}$ and $\eta^{right}$ bins (i,j).
A minimisation is performed to take into account the response measurements between many regions, to extract the inter-calibration factors for a specific $\eta$ region.
The inverse of the variance on each measured relative response, $\Delta\mathcal{R_{\rm ijk}}$, is used to weight the equations in a minimisation equation,
\begin{equation}
\sum\limits^N_{j=1} \sum\limits^N_{i=j}\left\{\frac{1}{\Delta\mathcal{R_{\rm ijk}}}( c_{jk}\mathcal{R_{\rm ijk}}- c_{ik} )\right\}^2 + X(c_{ik}).
\label{JetPerf:MM_Final}
\end{equation}

The first term in Equation \ref{JetPerf:MM_Final} is minimised to find the values of $c_{ik}$ that best agree with the measured $\mathcal{R_{\rm ijk}}$. 
A trivial undesired solution is $c_i=0$. 
The second term in \ref{JetPerf:MM_Final} is added to prevent this solution.
The specific form is 
\begin{equation}
X(c_{ik})= K(N^{-1}_{bins} \sum\limits^{N_{bins}}_{i=1} c_{ik} - 1 )^2,
\end{equation}
where K is a constant.
This term is a minimum when the average correction factor is equal to unity.
The constant, K, is set to $10^6$, and its purpose is to penalise deviations of the average away from 1. 
For large values of K, the correction factors found are stable.
Once the minimised $c_{ik}$ values are found, these values are rescaled such that the correction factors at $|\eta|<1$ are equal to unity. 

The advantage of this method with respect to the standard method is that each relative response is calculated using every $\eta$ bin combination, which gives an increase in the statistics used, especially at larger $\eta$, when compared to the standard method which needed one of the jets to be in the central probe region.
