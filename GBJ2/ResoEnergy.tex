\subsection{Jet Energy Resolution}
\label{GBJ2:JER}

The effect from the jet energy resolution (JER) was assessed by smearing the \pt{} and energy in every jet in an event using a Gaussian function with the width given by the measured JER, as described in Section XXX, of the jet at that \pt{} and $\eta$. 
The analysis was then rerun on the smeared jets and the ratio of the final plots were used to see the effect compared to the unsmeared sample.
The smearing procedure was repeated ten times and an average of all curves is the final effect of JER.
This averaging reduced the effect of statistical fluctuations in some bins.
The rapidity weighted reconstructed PYTHIA sample was used due to the high statistics in most regions.
This method assesses the effect of an increase in JES. 
Reducing the JER is not possible, so an symmetric uncertainty band is constructed from the increased smearing.

Figures \ref{GBJ2:ResoEnergy:Inclusive_gap} -- \ref{GBJ2:ResoEnergy:Q0} show the ratio of the final distributions from the smeared jets and from the nominal jets.
The blue histograms are the multiple smears, and the two red histogram are the uncertainty band found from the average of the blue histograms.  

Figure \ref{GBJ2:ResoEnergy:Inclusive_gap} (a) shows the ratio for the number of events as a function of the dijet rapidity separation, \dy{}.
The effect of the smearing is less than  $1\%$ in all bins.
Figure \ref{GBJ2:ResoEnergy:Inclusive_gap} (b) shows the ratio for the gap fraction as a function of the dijet rapidity separation, \dy{}.
The deviations from unity are very small for low \dy{} and even for large \dy{} are less than  $1\%$.


Figure \ref{GBJ2:ResoEnergy:cos} shows the ratio for average \cosdphi{} as a function of the dijet rapidity separation, \dy{}, for both (a) inclusive events and (b) gap events.
The JER smearing has a less than  $1\%$ effect on these distributions.
Figure \ref{GBJ2:ResoEnergy:cos2} shows the ratio for average \costwodphi{} as a function of the dijet rapidity separation, \dy{}, for both (a) inclusive events and (b) gap events.
The effects from the JER smearing are less than  $1\%$ for both the inclusive and gap events.

Figure \ref{GBJ2:ResoEnergy:dphi23} shows the ratio for the \dphi{} distribution for (a) inclusive events and (b) gap events with a \dy{} in the range 2--3.
The effect of the JES smearing is less than  $1\%$ for all bins.
For the lowest \dphi{} bins, the individual ratios of the smeared jets vary by more than at the high \dphi{} bins due to the statistical uncertainty being significantly larger.
Figure \ref{GBJ2:ResoEnergy:dphi45} shows the ratio for the \dphi{} distribution for (a) inclusive events and (b) gap events with a \dy{} in the range 4--5.
For both the inclusive and gap events, the effect of the JER smearing is at maximum  $1\%$.
Figure \ref{GBJ2:ResoEnergy:dphi78} shows the ratio for the \dphi{} distribution for (a) inclusive events and (b) gap events with a \dy{} in the range 7--8.
For the inclusive events, the effect of the JER smearing is of the order of $2\%$.
For the gap events, the effect of the JER smearing can get large, especially in the low \dphi{} bins. 
Ignoring the two lowest \dphi{} bins, which do not have any data events in them, the uncertainty does not exceed $2\%$.

Figure \ref{GBJ2:ResoEnergy:Q0} shows the ratio for the gap fraction as a function of \qz{} for the \dy{} ranges (a) 2--3, (b) 4--5 and (c) 7--8. 
The effect from the smearing is less than $1\%$ in all the ranges.
The spread in the individual smears is larger for more separated dijets due to the increase in statistical uncertainty. 


\begin{figure}
\centering
\mbox{
              \subfigure[]{\epsfig{figure=figures/GBJ2/ResoEnergy/RMS_E___Nevent_deltaY_Ratio.eps,width=0.5\textwidth}}\quad
              \subfigure[]{\epsfig{figure=figures/GBJ2/ResoEnergy/RMS_E___GapFraction_deltaY_Ratio.eps,width=0.5\textwidth}}\quad
                              }
\caption[]{
The ratio of (a) inclusive distribution and (b) gap fraction as a function of $\Delta y$ for reconstructed PYTHIA sample with nominal sample compared to energy smeared sample.
\label{GBJ2:ResoEnergy:Inclusive_gap}}
\end{figure}

\begin{figure}
\centering
\mbox{
              \subfigure[]{\epsfig{figure=figures/GBJ2/ResoEnergy/RMS_E___cosdPhi_deltaY_Ratio.eps,width=0.5\textwidth}}\quad
              \subfigure[]{\epsfig{figure=figures/GBJ2/ResoEnergy/RMS_E___cosdPhi_deltaY_gap_Ratio.eps,width=0.5\textwidth}}\quad
                              }
\caption[]{
The ratio of average \cosdphi{} as a function of $\Delta y$ for (a) inclusive and (b) gap events reconstructed PYTHIA sample with nominal sample compared to energy smeared sample.
\label{GBJ2:ResoEnergy:cos}}
\end{figure}

\begin{figure}
\centering
\mbox{
              \subfigure[]{\epsfig{figure=figures/GBJ2/ResoEnergy/RMS_E___cos2dPhi_deltaY_Ratio.eps,width=0.5\textwidth}}\quad
              \subfigure[]{\epsfig{figure=figures/GBJ2/ResoEnergy/RMS_E___cos2dPhi_deltaY_gap_Ratio.eps,width=0.5\textwidth}}\quad
                              }
\caption[]{
The ratio of average \costwodphi{} as a function of $\Delta y$ for (a) inclusive and (b) gap events reconstructed PYTHIA sample with nominal sample compared to energy smeared sample.
\label{GBJ2:ResoEnergy:cos2}}
\end{figure}


\begin{figure}
\centering
\mbox{
              \subfigure[]{\epsfig{figure=figures/GBJ2/ResoEnergy/RMS_E___dPhi__2_3_Ratio.eps,width=0.5\textwidth}}\quad
              \subfigure[]{\epsfig{figure=figures/GBJ2/ResoEnergy/RMS_E___dPhi_gap__2_3_Ratio.eps,width=0.5\textwidth}}\quad
                              }
\caption[]{
The ratio of the \dphi{} distribution for $\Delta y$ of 2--3 for (a) inclusive and (b) gap events reconstructed PYTHIA sample with nominal sample compared to energy smeared sample.
\label{GBJ2:ResoEnergy:dphi23}}
\end{figure}


\begin{figure}
\centering
\mbox{
              \subfigure[]{\epsfig{figure=figures/GBJ2/ResoEnergy/RMS_E___dPhi__4_5_Ratio.eps,width=0.5\textwidth}}\quad
              \subfigure[]{\epsfig{figure=figures/GBJ2/ResoEnergy/RMS_E___dPhi_gap__4_5_Ratio.eps,width=0.5\textwidth}}\quad
                              }
\caption[]{
The ratio of the \dphi{} distribution for $\Delta y$ of 4--5 for (a) inclusive and (b) gap events reconstructed PYTHIA sample with nominal sample compared to energy smeared sample.
\label{GBJ2:ResoEnergy:dphi45}}
\end{figure}



\begin{figure}
\centering
\mbox{
              \subfigure[]{\epsfig{figure=figures/GBJ2/ResoEnergy/RMS_E___dPhi__7_8_Ratio.eps,width=0.5\textwidth}}\quad
              \subfigure[]{\epsfig{figure=figures/GBJ2/ResoEnergy/RMS_E___dPhi_gap__7_8_Ratio.eps,width=0.5\textwidth}}\quad
                              }
\caption[]{
The ratio of the \dphi{} distribution for $\Delta y$ of 7--8 for (a) inclusive and (b) gap events reconstructed PYTHIA sample with nominal sample compared to energy smeared sample.
\label{GBJ2:ResoEnergy:dphi78}}
\end{figure}


\begin{figure}
\centering
\mbox{
              \subfigure[]{\epsfig{figure=figures/GBJ2/ResoEnergy/RMS_E_2_3__Q0_Ratio.eps,width=0.33\textwidth}}
              \subfigure[]{\epsfig{figure=figures/GBJ2/ResoEnergy/RMS_E_4_5__Q0_Ratio.eps,width=0.33\textwidth}}
              \subfigure[]{\epsfig{figure=figures/GBJ2/ResoEnergy/RMS_E_7_8__Q0_Ratio.eps,width=0.33\textwidth}}
                              }
\caption[]{
The gap fraction as a function of \qz{} for dijet events with a  \dy{} of (a) 2--3 (b) 4--5 and (c) 7--8 for reconstructed PYTHIA sample with nominal sample compared to energy smeared sample.
\label{GBJ2:ResoEnergy:Q0}}
\end{figure}



