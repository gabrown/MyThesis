\section{Topology Selection}
\label{sec:GBJ2:AnalSel}
The jets used in this analysis were reconstructed from EM scale calorimter clusters with the anti$-k_T$ algorithm, described in Section \ref{sec:Theory:Jets}, using a radius parameter R=0.6.
As for the previous analysis, jets that have \pt{}$>20$ GeV and $|y|<4.4$ are used, as these have a well defined jet energy scale and jet cleaning cuts.
The dijet system is defined by the two leading jets in the event, with cuts on the transverse momentum of the leading and sub-leading jets of $\pt{}>60$ GeV and $\pt{}>50$ GeV respectively.



The observables studied can loosely be divided into azimuthal decorrelation observables, and jet veto observables.
The jet veto observables are the same as defined in the previous chapter, where the gap fraction, \gap{}, is defined by,
\begin{equation}
f_{\rm gap} = \frac{n (\qz{})}{N},
\label{GBJ2:fgap}
\end{equation}
where n(\qz{}) is the number of gap events which pass the jet veto, with scale \qz{}, and $N$ is the inclusive number of events.
The gap fraction is studied as a function of \dy{} and \qz{}.
The studied is \nb{}, which is the mean number of jets found in the rapidity region between the dijets \nb{} that have \pt{}$>$\qz{}. 
The azimuthal decorrelation, \dphi{}, cross section is measured in bins of \dy{} for inclusive events and events that have survived a jet veto.
The average cosine of the \dphi{}, $\mean{\cosdphi{}}$ and the average of twice \dphi{}, $\mean{\costwodphi{}}$ is studied as a function of \dy{} for inclusive events and events which survive a jet veto.

When slices in \dy{} are presented, three standard slices have been chosen to probe \dy{} of 2--3, 4--5, and 7--8.
For the final plots, all \dy{} slices from 0--8. 
The standard jet veto scale that is used to define gap events is $\qz{}=20$ GeV.


