\section{Topology Selection}
\label{sec:GBJ2:AnalSel}
The jets used in this analysis were reconstructed from EM scale calorimeter clusters with the anti$-k_T$ algorithm, described in Section \ref{sec:Theory:Jets}, using a radius parameter R=0.6.
As for the previous analysis, jets that have \pt{}$>20$ GeV and $|y|<4.4$ are used, as these have a well defined jet energy scale and jet cleaning cuts.
The dijet system is defined by the two leading jets in the event, with cuts on the transverse momentum of the leading and sub-leading jets of $\pt{}>60$ GeV and $\pt{}>50$ GeV, respectively.



The observables studied can loosely be divided into azimuthal decorralation observables, and jet veto observables.
The jet veto observables are the same as defined in the previous chapter, where the gap fraction, \gap{}, is defined by,
\begin{equation}
f_{\rm gap} = \frac{n (\qz{})}{N},
\label{GBJ2:fgap}
\end{equation}
where n(\qz{}) is the number of gap events which pass the jet veto, with scale \qz{}, and $N$ is the inclusive number of events.
The gap fraction is studied as a function of \dy{} and \qz{}.
Again the mean number of jets found in the rapidity region between the dijets \nb{} that have \pt{}$>$\qz{} is also studied. 
The azimuthal decorralation variables are
\begin{equation}
\frac{d\sigma}{d\Delta\phi},~\mean{\cosdphi}~\rm{and}~\mean{\costwodphi}.
\label{GBJ2:AZVar}
\end{equation}
The cross section as a function of \dphi{} is measured in slices of \dy{}.
The average decorrelation variables (\mean{\cosdphi} and \mean{\costwodphi})  are studied as a function of \dy{}
All decorrelation variables are studied separately for inclusive and gap events (those that survive a jet veto).
The standard jet veto scale that is used to define gap events is $\qz{}=20$ GeV.


