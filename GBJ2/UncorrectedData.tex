\section{Assessment of Uncorrected Data vs MC}
\label{sec:GBJ2:Uncorr}
This section presents the data compared to the reconstructed PYTHIA sample.

Figures \ref{GBJ2:Uncorr:Incl_Gap} -- \ref{GBJ2:Uncorr:Q0} show the uncorrected data compared to PYTHIA.
The PYTHIA error bands are the quadrature sum of the statistical error and the JES uncertainty bands. 

Figure \ref{GBJ2:Uncorr:Incl_Gap} shows (a) the gap fraction and (b) the mean number of jets in the region bounded by the dijet system as a function of \dy{}.
The gap fraction from data reduces as a function of \dy{}, but for $\dy{}>5.5$ it starts to level off.
The PYTHIA gap fraction curve is consistantly below the data until a $\dy{}=6.5$.
PYTHIA then rises for the larges \dy{} bins, and has a higher hap fraction for $\dy{}>7$. 
The mean multiplicity of jets in the rapidity region increases as a function of \dy{}, to a peak of $\sim1.2$ at $\dy{}=7$, and then plateaus.


Figures \ref{GBJ2:Uncorr:dphi23}, \ref{GBJ2:Uncorr:dphi45}, and \ref{GBJ2:Uncorr:dphi78} show \dphiDist{} for $2<\dy{}<3$, $4<\dy{}<5$, and $7<\dy{}<8$,  respectively, for (a) inclusive events and (b) gap events.
The agreement of the cross
In these \dphi{} plots the agreement is not quite as good especially at high \dphi{} for  $2<\dy{}<3$ and $4<\dy{}<5$ where there is a difference of about $20\%$. 
It is not surprising that the cross section from data does not agree with PYTHIA cross section, as there are large uncertainties on the leading order + leading logarithm cross section from PYTHIA.

Figures \ref{GBJ2:Uncorr:cos} and \ref{GBJ2:Uncorr:cos2} show the \mean{\cosdphi{}} and \mean{\costwodphi{}} as a function of the dijet separation, \dy{} for (a) inclusive and (b) gap events.
A \mean{\cosdphi{}} value of one corresponds to perfectly back-to-back in azimuth jets.
The \mean{\cosdphi{}} distribution for the inclusive events shows a value of $ \mean{\cosdphi{}}\sim0.95$ for $\dy{}<2$.
As the $\dy{}$ increases the jets become less back-to-back and the \mean{\cosdphi{}} falls and stays at a value of $ \mean{\cosdphi{}}\sim0.86$ for $\dy{}=6$.
As the $\dy{}$ increase the available phase-space to emit into is larger due to the jet being at very high energies.
For the gap events, the mean{\cosdphi{}} at low \dy{} starts at a similar level to the inclusive events, but then slowly rises to a maximum of $\sim 0.97$.
When the \dy{} is low, emissions can fall outside the rapidity region, but as the \dy{} increases this region becomes smaller, and the jet veto is stopping hard emission into the rapidity region, thus the dijets are more back-to-back.
The PYTHIA has a slightly different shape than the data.
At both low and high \dy{}, the \mean{\cosdphi{}} for PYTHIA is higher than the data for both gap and inclusive events. 
In the range $2<\dy{}<5.5$, PYTHIA describes the data well.
The shape of the \mean{\costwodphi{}} distribution has a similar explaination to the \mean{\cosdphi{}} distribution, and shows similar features.
The value of \mean{\costwodphi{}} at $\dy{}=0$ is 0.8.
This then falls to $\mean{\costwodphi{}}=0.6$
********Add some more stuff here*********

Figure \ref{GBJ2:Uncorr:Q0} show the gap fraction as a function of \qz{}, for the \dy{} ranges $2<\dy{}<3$, $4<\dy{}<5$, and $7<\dy{}<8$.
There are some differences between the PYTHIA sample and uncorrected data, especially in the low \qz{} in the 2--3 \dy{} slice.


\begin{figure}
\centering
\mbox{
              \subfigure[]{\epsfig{figure=figures/GBJ2/ControlPlots/Smeared__GapFraction_deltaY.eps,width=0.5\textwidth}}\quad
              \subfigure[]{\epsfig{figure=figures/GBJ2/ControlPlots/Smeared__prof_deltaY_njets.eps,width=0.5\textwidth}}\quad
                              }
\caption[]{
(a) The inclusive distribution and (b) the gap fraction against $\Delta y$ for 2010 uncorrected data (black points) and reconstructed PYTHIA sample (red points).
\label{GBJ2:Uncorr:Incl_Gap}}
\end{figure}



\begin{figure}
\centering
\mbox{
              \subfigure[]{\epsfig{figure=figures/GBJ2/ControlPlots/Smeared__dPhi__2_3.eps,width=0.5\textwidth}}\quad
              \subfigure[]{\epsfig{figure=figures/GBJ2/ControlPlots/Smeared__dPhi_gap__2_3.eps,width=0.5\textwidth}}\quad
                              }
\caption[]{
\dphi{} distribution for (a) inclusive events and (b) gap events for a dijet separation, \dy{} of 2-3 for 2010 uncorrected data (black points) and reconstructed PYTHIA sample (red points).
\label{GBJ2:Uncorr:dphi23}}
\end{figure}


\begin{figure}
\centering
\mbox{
              \subfigure[]{\epsfig{figure=figures/GBJ2/ControlPlots/Smeared__dPhi__4_5.eps,width=0.5\textwidth}}\quad
              \subfigure[]{\epsfig{figure=figures/GBJ2/ControlPlots/Smeared__dPhi_gap__4_5.eps,width=0.5\textwidth}}\quad
                              }
\caption[]{
\dphi{} distribution for (a) inclusive events and (b) gap events for a dijet separation, \dy{} of 4-5 for 2010 uncorrected data (black points) and reconstructed PYTHIA sample (red points).
\label{GBJ2:Uncorr:dphi45}}
\end{figure}



\begin{figure}
\centering
\mbox{
              \subfigure[]{\epsfig{figure=figures/GBJ2/ControlPlots/Smeared__dPhi__7_8.eps,width=0.5\textwidth}}\quad
              \subfigure[]{\epsfig{figure=figures/GBJ2/ControlPlots/Smeared__dPhi_gap__7_8.eps,width=0.5\textwidth}}\quad
                              }
\caption[]{
\dphi{} distribution for (a) inclusive events and (b) gap events for a dijet separation, \dy{} of 7-8 for 2010 uncorrected data (black points) and reconstructed PYTHIA sample (red points).
\label{GBJ2:Uncorr:dphi78}}
\end{figure}



\begin{figure}
\centering
\mbox{
              \subfigure[]{\epsfig{figure=figures/GBJ2/ControlPlots/Smeared__cosdPhi_deltaY.eps,width=0.5\textwidth}}\quad
              \subfigure[]{\epsfig{figure=figures/GBJ2/ControlPlots/Smeared__cosdPhi_deltaY_gap.eps,width=0.5\textwidth}}\quad
                              }
\caption[]{
\mean{\cosdphi{}} as a function of \dy{} for (a) inclusive and (b) gap events for 2010 uncorrected data (black points) and reconstructed PYTHIA sample (red points).
\label{GBJ2:Uncorr:cos}}
\end{figure}


\begin{figure}
\centering
\mbox{
              \subfigure[]{\epsfig{figure=figures/GBJ2/ControlPlots/Smeared__cos2dPhi_deltaY.eps,width=0.5\textwidth}}\quad
              \subfigure[]{\epsfig{figure=figures/GBJ2/ControlPlots/Smeared__cos2dPhi_deltaY_gap.eps,width=0.5\textwidth}}\quad
                              }
\caption[]{
\mean{\costwodphi{}} as a function of \dy{} for (a) inclusive and (b) gap events for 2010 uncorrected data (black points) and reconstructed PYTHIA sample (red points).
\label{GBJ2:Uncorr:cos2}}
\end{figure}

\begin{figure}
\centering
\mbox{
      \subfigure[]{\epsfig{figure=figures/GBJ2/ControlPlots/Smeared2_3__Q0,width=0.5\textwidth}}\quad
      \subfigure[]{\epsfig{figure=figures/GBJ2/ControlPlots/Smeared4_5__Q0,width=0.5\textwidth}}\quad
}
\mbox{
      \subfigure[]{\epsfig{figure=figures/GBJ2/ControlPlots/Smeared7_8__Q0,width=0.5\textwidth}}\quad
}
\caption[]{
The gap fraction against \qz{} for \dy{} ranges (a) 2--3, (b) 4--5 and (c) 7--8 for 2010 uncorrected data (black points) and reconstructed PYTHIA sample (red points).
\label{GBJ2:Uncorr:Q0}}
\end{figure}

