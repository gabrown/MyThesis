\section{Assesment of Uncorrected Data vs MC}
\label{sec:GBJ2:Uncorr}
This section presents the data compared to the reconstructed PYTHIA sample, to check that the PYTHIA sample agrees with data sufficiently well to enable it to be used for unfolding the detector effects in the data.

Figures \ref{GBJ2:Uncorr:Incl_Gap} -- \ref{GBJ2:Uncorr:Q0} show the uncorrected data compared to PYTHIA.
The PYTHIA error bands are the quadrature sum of the statistical error and the JES uncertainty bands. 

Figure \ref{GBJ2:Uncorr:Incl_Gap} shows (a) the inclusive distribution and (b) the gap fraction as a function of \dy{}.
In the inclusive distribution the PYTHIA is consistently below the data, but are within the uncertainty bands.
Furthermore, the PYTHIA has the correct shape, even if the normalisation is slightly wrong.
In the gap fraction plot, PYTHIA has a slightly lower gap fraction except at the highest \dy{} bins, but the points are not too far from the data.

Figures \ref{GBJ2:Uncorr:dphi23}, \ref{GBJ2:Uncorr:dphi45}, and \ref{GBJ2:Uncorr:dphi78} show the \dphi{} distribution for \dy{} ranges of 2--3, 4--5, and 7--8 respectively, for (a) inclusive events and (b) gap events.
In these \dphi{} plots the agreement is not quite as good especially at high \dphi{} for \dy{} of 2--3 and 4--5, where there is a difference of about $20\%$. 

Figures \ref{GBJ2:Uncorr:cos} and \ref{GBJ2:Uncorr:cos2} show the average \cosdphi{} and average \costwodphi{} as a function of the dijet separation, \dy{} for (a) inclusive and (b) gap events.
In general, PYTHIA agrees well with all the distributions quite well, but there are some small deviations in the largets \dy{} bins. 

Figure \ref{GBJ2:Uncorr:Q0} show the gap fraction as a function of \qz{}, for the \dy{} ranges (a) 2--3, (b) 4--5 and (c) 7--8.
There are some differences between the PYTHIA sample and uncorrected data, especially in the low \qz{} in the 2--3 \dy{} slice.


\begin{figure}
\centering
\mbox{
              \subfigure[]{\epsfig{figure=figures/GBJ2/ControlPlots/Smeared__Nevent_deltaY.eps,width=0.5\textwidth}}\quad
              \subfigure[]{\epsfig{figure=figures/GBJ2/ControlPlots/Smeared__GapFraction_deltaY.eps,width=0.5\textwidth}}\quad
                              }
\caption[]{
(a) The inclusive distribution and (b) the gap fraction against $\Delta y$ for 2010 uncorrected data (black points) and reconstructed PYTHIA sample (red points).
\label{GBJ2:Uncorr:Incl_Gap}}
\end{figure}



\begin{figure}
\centering
\mbox{
              \subfigure[]{\epsfig{figure=figures/GBJ2/ControlPlots/Smeared__dPhi__2_3.eps,width=0.5\textwidth}}\quad
              \subfigure[]{\epsfig{figure=figures/GBJ2/ControlPlots/Smeared__dPhi_gap__2_3.eps,width=0.5\textwidth}}\quad
                              }
\caption[]{
\dphi{} distribution for (a) inclusive events and (b) gap events for a dijet separation, \dy{} of 2-3 for 2010 uncorrected data (black points) and reconstructed PYTHIA sample (red points).
\label{GBJ2:Uncorr:dphi23}}
\end{figure}


\begin{figure}
\centering
\mbox{
              \subfigure[]{\epsfig{figure=figures/GBJ2/ControlPlots/Smeared__dPhi__4_5.eps,width=0.5\textwidth}}\quad
              \subfigure[]{\epsfig{figure=figures/GBJ2/ControlPlots/Smeared__dPhi_gap__4_5.eps,width=0.5\textwidth}}\quad
                              }
\caption[]{
\dphi{} distribution for (a) inclusive events and (b) gap events for a dijet separation, \dy{} of 4-5 for 2010 uncorrected data (black points) and reconstructed PYTHIA sample (red points).
\label{GBJ2:Uncorr:dphi45}}
\end{figure}



\begin{figure}
\centering
\mbox{
              \subfigure[]{\epsfig{figure=figures/GBJ2/ControlPlots/Smeared__dPhi__7_8.eps,width=0.5\textwidth}}\quad
              \subfigure[]{\epsfig{figure=figures/GBJ2/ControlPlots/Smeared__dPhi_gap__7_8.eps,width=0.5\textwidth}}\quad
                              }
\caption[]{
\dphi{} distribution for (a) inclusive events and (b) gap events for a dijet separation, \dy{} of 7-8 for 2010 uncorrected data (black points) and reconstructed PYTHIA sample (red points).
\label{GBJ2:Uncorr:dphi78}}
\end{figure}



\begin{figure}
\centering
\mbox{
              \subfigure[]{\epsfig{figure=figures/GBJ2/ControlPlots/Smeared__cosdPhi_deltaY.eps,width=0.5\textwidth}}\quad
              \subfigure[]{\epsfig{figure=figures/GBJ2/ControlPlots/Smeared__cosdPhi_deltaY_gap.eps,width=0.5\textwidth}}\quad
                              }
\caption[]{
Average \cosdphi{} as a function of \dy{} for (a) inclusive and (b) gap events for 2010 uncorrected data (black points) and reconstructed PYTHIA sample (red points).
\label{GBJ2:Uncorr:cos}}
\end{figure}


\begin{figure}
\centering
\mbox{
              \subfigure[]{\epsfig{figure=figures/GBJ2/ControlPlots/Smeared__cos2dPhi_deltaY.eps,width=0.5\textwidth}}\quad
              \subfigure[]{\epsfig{figure=figures/GBJ2/ControlPlots/Smeared__cos2dPhi_deltaY_gap.eps,width=0.5\textwidth}}\quad
                              }
\caption[]{
Average \costwodphi{} as a function of \dy{} for (a) inclusive and (b) gap events for 2010 uncorrected data (black points) and reconstructed PYTHIA sample (red points).
\label{GBJ2:Uncorr:cos2}}
\end{figure}

\begin{figure}
\centering
\mbox{
      \subfigure[]{\epsfig{figure=figures/GBJ2/ControlPlots/Smeared2_3__Q0,width=0.5\textwidth}}\quad
      \subfigure[]{\epsfig{figure=figures/GBJ2/ControlPlots/Smeared4_5__Q0,width=0.5\textwidth}}\quad
}
\mbox{
      \subfigure[]{\epsfig{figure=figures/GBJ2/ControlPlots/Smeared7_8__Q0,width=0.5\textwidth}}\quad
}
\caption[]{
The gap fraction against \qz{} for \dy{} ranges (a) 2--3, (b) 4--5 and (c) 7--8 for 2010 uncorrected data (black points) and reconstructed PYTHIA sample (red points).
\label{GBJ2:Uncorr:Q0}}
\end{figure}

