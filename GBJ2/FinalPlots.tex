\section{Corrected Data}
\label{sec:GBJ2:FinalPlots}

First, the corrected data, with the systematic uncertainties discussed, is first compared to some common MC generators in Figures \ref{GBJ2:FinalPlots:GapFracLO} and \ref{GBJ2:FinalPlots:NJetLO}.
The MC generators chosen for the comparison are PYTHIA, HERWIG++ and SHERPA. 

Figure \ref{GBJ2:FinalPlots:GapFracLO} shows (a) the gap fraction against \dy{} for the corrected data and the MC generators, and (b) shows the ratio of the MCs to the data.
In general, PYTHIA gives slightly too low a gap fraction at low \dy{} but improves at larger \dy{}. 
HERWIG does quite well at low \dy{} but from \dy{} of 4 onwards it diverges from the data giving too low a gap fraction. 
SHERPA agrees quite well with the data, but better at large \dy{}

Figure \ref{GBJ2:FinalPlots:NJetLO} shows (a) the average number of jets in the dijet rapidity region against \dy{} for the corrected data and the MC generators, and (b) shows the ratio of the MCs to the data.
As with the gap fraction, PYTHIA agrees with the data quite well, but underestimates the data at large \dy{}.
HERWIG agrees with the data at low \dy{}, but diverges at larger \dy{}.
SHERPA on the whole agrees well with the data, but around a \dy{} of 4--7 underestimates the number of jets in the rapidity region.

The corrected data is compared the theory predictions from HEJ and POWHEG. 
The theory predictions are presented as a band to represent the theoretical uncertainty, which was found by varying both the renormalisation and factorisation scale and also the choice of PDF.
Two POWHEG curves are shown which have the different parton showering, hadronization and underlying event of PYTHIA and HERWIG. 
The HEJ events have been passed through Ariadne which ....


Figure \ref{GBJ2:FinalPlots:GapFrac} shows (a) the gap fraction against \dy{} for the corrected data and the theoretical predictions, and (b) shows the ratio of the predictions to the data.
There is significant differences between the three theoretical predictions and the data.
POWHEG + PYTHIA gives the best agreement with data, and while the gap fraction is a bit low at large \dy{}, it does have the plateau that is observed in the data. 
Conversely, POWHEG + HERWIG has too low a gap fraction from from about a \dy{} of 2.
The HEJ prediction also gives too low a gap fraction, but not as far from the data as the POWHEG + HERWIG curve.

Figure \ref{GBJ2:FinalPlots:nJets} shows (a) the average number of jets in the dijet rapidity region against \dy{} for the corrected data and the theoretical predictions, and (b) shows the ratio of the predictions to the data.
HEJ agrees with the data well until a \dy{} of 4, but then gives too many jets, and doesn't replicate the plateau observed in the data. 
POWHEG + HERWIG only agrees with data at the very lowest \dy{} bins, then has too many jets in the rapidity region for larger \dy{} and doesn't have the plateau either.
POWHEG + PYTHIA agrees well best with the data, agreeing in all \dy{} bins and has a very good agreement at the plateau.

Figure \ref{GBJ2:FinalPlots:Q0} shows (a) the gap fraction as a function of \qz{} for the three standard \dy{} slices for the corrected data and the theoretical predictions, and (b) shows the ratio of the predictions to the data.
All the theoretical predictions agree with data at large \qz{} where the gap fraction is one, this is because the \qz{} will, most of the time, be set above the \pt{} of the leading jets, so there cannot be any other jets in the event with \pt{} greater than \qz{}
HEJ gives too low a gap fraction in the lower \qz{} region, especially for the higher \dy{} ranges, which agrees with the gap fraction against \dy{} in Figure \ref{GBJ2:FinalPlots:GapFrac}.
POWHEG + HERWIG has too low a gap fraction at the low \qz{} region for all \dy{} slices, but as with HEJ, it get worse at larger \dy{} ranges.
POWHEG + PYTHIA gives the best description of the data. In slice ... (need to see ratio)
********Something is wrong with this plot need to look at division.**********


Figure \ref{GBJ2:FinalPlots:dphi_Incl} shows (a) the differential cross section as a function of \dphi{} for the three standard \dy{} slices for the corrected data and the theoretical predictions, and (b) shows the ratio of the predictions to the data.
For the low \dy{} slice, all the theoretical predictions poorly match the data at low \dphi{}, but improve for larger \dphi{}.
HEJ does particularly well in the 4--5 \dy{} slice, agreeing with the data for all \dphi{}.
Both POWHEG + HERWIG and POWHEG + PYTHIA have a larger cross section for all but the largest two \dphi{} bins for the 4--5 \dy{} slice.
In the 7--8 \dy{} slice, both POWHEG + HERWIG and POWHEG + PYTHIA agree with the data within the systematic uncertainty band, which is quite large in this slice.
HEJ how too low a cross section in the 7--8 \dy{} slice. 



\begin{figure}
\centering
\mbox{
              \subfigure[]{\epsfig{figure=figures/GBJ2/FinalPlots/LO_GapFraction_dyBins.eps,width=0.5\textwidth}}\quad
              \subfigure[]{\epsfig{figure=figures/GBJ2/FinalPlots/LO_GapFraction_dyBins_Ratio.eps,width=0.5\textwidth}}\quad
                              }
\caption[]{
(a) The gap fraction as a function of \dy{}.  
The 2010 corrected data (black points) is shown with statistical (black error bars) and systematic (yellow error band) error bands against leading order MCs of PYTHIA, HERWIG++ and SHERPA.
The ratio to the data for (a)  is shown in (b).
\label{GBJ2:FinalPlots:GapFracLO}}
\end{figure}

\begin{figure}
\centering
\mbox{
              \subfigure[]{\epsfig{figure=figures/GBJ2/FinalPlots/LO_nGapJets_dyBins.eps,width=0.5\textwidth}}\quad
              \subfigure[]{\epsfig{figure=figures/GBJ2/FinalPlots/LO_nGapJets_dyBins_Ratio.eps,width=0.5\textwidth}}\quad
                              }
\caption[]{
(a) The average number of jets in the dijet rapidity region as a function of \dy{}.  
The 2010 corrected data (black points) is shown with statistical (black error bars) and systematic (yellow error band) error bands against leading order MCs of PYTHIA, HERWIG++ and SHERPA.
The ratio to the data for (a) is shown in (b).
\label{GBJ2:FinalPlots:NJetLO}}
\end{figure}

\begin{figure}
\centering
\mbox{
              \subfigure[]{\epsfig{figure=figures/GBJ2/FinalPlots/GapFraction_dyBins.eps,width=0.5\textwidth}}\quad
              \subfigure[]{\epsfig{figure=figures/GBJ2/FinalPlots/GapFraction_dyBins_Ratio.eps,width=0.5\textwidth}}\quad
                              }
\caption[]{
(a) The gap fraction as a function of \dy{}.  
The 2010 corrected data (black points) is shown with statistical (black error bars) and systematic (yellow error band) error bands against HEJ and POWHEG predictions.
The ratio of the theoretical predictions to the data is shown in (b).
\label{GBJ2:FinalPlots:GapFrac}}
\end{figure}



\begin{figure}
\centering
\mbox{
              \subfigure[]{\epsfig{figure=figures/GBJ2/FinalPlots/nGapJets_dyBins.eps,width=0.5\textwidth}}\quad
              \subfigure[]{\epsfig{figure=figures/GBJ2/FinalPlots/nGapJets_dyBins_Ratio.eps,width=0.5\textwidth}}\quad
                              }
\caption[]{
(a) The average number of jets in the dijet rapidity region as a function of \dy{}.  
The 2010 corrected data (black points) is shown with statistical (black error bars) and systematic (yellow error band) error bands against HEJ and POWHEG predictions.
The ratio to the data for (a) is shown in (b). 
\label{GBJ2:FinalPlots:nJets}}
\end{figure}

\begin{figure}
\centering
\mbox{
              \subfigure[]{\epsfig{figure=figures/GBJ2/FinalPlots/GapFraction_Q0.eps,width=0.5\textwidth}}\quad
              \subfigure[]{\epsfig{figure=figures/GBJ2/FinalPlots/GapFraction_Q0_Ratio.eps,width=0.5\textwidth}}\quad
                              }
\caption[]{
The gap fraction as a function of \qz{} for the three standard \dy{} ranges.  
The 2010 corrected data (black points) is shown with statistical (black error bars) and systematic (yellow error band) error bands against HEJ and POWHEG predictions.
The ratio of the theoretical predictions to the data is shown in (b).
\label{GBJ2:FinalPlots:Q0}}
\end{figure}

\begin{figure}
\centering
\mbox{
              \subfigure[]{\epsfig{figure=figures/GBJ2/FinalPlots/CrossSection.Inclusive.DPhiBins.AlldY.eps,width=0.5\textwidth}}\quad
              \subfigure[]{\epsfig{figure=figures/GBJ2/FinalPlots/CrossSection.Inclusive.DPhiBins.AlldY.Ratio.eps,width=0.5\textwidth}}\quad
                              }
\caption[]{
The differential cross section as a function of \dphi{} for inclusive events for the three standard \dy{} ranges.  
The 2010 corrected data (black points) is shown with statistical (black error bars) and systematic (yellow error band) error bands against HEJ and POWHEG predictions.
The ratio of the theoretical predictions to the data is shown in (b).
\label{GBJ2:FinalPlots:dPhi_Incl}}
\end{figure}

\begin{figure}
\centering
\mbox{
              \subfigure[]{\epsfig{figure=figures/GBJ2/FinalPlots/CrossSection.Gap.DPhiBins.AlldY.eps,width=0.5\textwidth}}\quad
              \subfigure[]{\epsfig{figure=figures/GBJ2/FinalPlots/CrossSection.Gap.DPhiBins.AlldY.Ratio.eps,width=0.5\textwidth}}\quad
                              }
\caption[]{
The differential cross section as a function of \dphi{} for inclusive events for the three standard \dy{} ranges.  
The 2010 corrected data (black points) is shown with statistical (black error bars) and systematic (yellow error band) error bands against HEJ and POWHEG predictions.
The ratio of the theoretical predictions to the data is shown in (b).
\label{GBJ2:FinalPlots:dPhi_Gap}}
\end{figure}


\begin{figure}
\centering
\mbox{
              \subfigure[]{\epsfig{figure=figures/GBJ2/FinalPlots/CosDeltaPhiInclusive_dyBins.eps,width=0.5\textwidth}}\quad
              \subfigure[]{\epsfig{figure=figures/GBJ2/FinalPlots/CosDeltaPhiInclusive_dyBins_Ratio.eps,width=0.5\textwidth}}\quad
                              }
\mbox{
              \subfigure[]{\epsfig{figure=figures/GBJ2/FinalPlots/CosDeltaPhiGap_dyBins.eps,width=0.5\textwidth}}\quad
              \subfigure[]{\epsfig{figure=figures/GBJ2/FinalPlots/CosDeltaPhiGap_dyBins_Ratio.eps,width=0.5\textwidth}}\quad
                              }
\caption[]{
The average \cosdphi{} as a function of \dy{} for (a) inclusive events and (c) gap events for the three standard \dy{} ranges.  
The 2010 corrected data (black points) is shown with statistical (black error bars) and systematic (yellow error band) error bands against HEJ and POWHEG predictions.
The ratio to the data for (a) and (c) is shown in (b) and (d) respectively.
\label{GBJ2:FinalPlots:Cos}}
\end{figure}

\begin{figure}
\centering
\mbox{
              \subfigure[]{\epsfig{figure=figures/GBJ2/FinalPlots/CosTwoDeltaPhiInclusive_dyBins.eps,width=0.5\textwidth}}\quad
              \subfigure[]{\epsfig{figure=figures/GBJ2/FinalPlots/CosTwoDeltaPhiInclusive_dyBins_Ratio.eps,width=0.5\textwidth}}\quad
                              }
\mbox{
              \subfigure[]{\epsfig{figure=figures/GBJ2/FinalPlots/CosTwoDeltaPhiGap_dyBins.eps,width=0.5\textwidth}}\quad
              \subfigure[]{\epsfig{figure=figures/GBJ2/FinalPlots/CosTwoDeltaPhiGap_dyBins_Ratio.eps,width=0.5\textwidth}}\quad
                              }
\caption[]{
The average \costwodphi{} as a function of \dy{} for (a) inclusive events and (c) gap events for the three standard \dy{} ranges.  
The 2010 corrected data (black points) is shown with statistical (black error bars) and systematic (yellow error band) error bands against HEJ and POWHEG predictions.
The ratio to the data for (a) and (c) is shown in (b) and (d) respectively.
\label{GBJ2:FinalPlots:Cos2}}
\end{figure}



