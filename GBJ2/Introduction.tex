The first LHC measurement of dijet production with a jet veto (documented in Chapter \ref{chp:GBJ1} and \cite{ref:ATLASGap}) examined the gap fraction in slices \ptb{} up to $\dy{}=6$.
This cut-off in \dy{} is necessary because the analysis uses a central jet trigger strategy in which the trigger fires if a jet is within $|y|<2.8$.
There is a relatively small number of events at large \dy{} where the statistical uncertainties are large. 
The data are compared to theory predictions by HEJ and POWHEG.

The analysis is extended to probe the gap fraction at a larger \dy{} region.
This larger \dy{} region is again studied using the 2010 data, but using a new dijet trigger strategy that allowes measurements up to $\dy{}=8$, which is the boundary of the detector.
In addition, the effect of quark/gluon emission from the dijet system is studied using the azimuthal decorrelation between the jets that make up the dijet system.
The previous ATLAS measurement of azimuthal decorrelation was carried out using jets with $|y| < 0.8$ \cite{ref:Decorr}.
Measurements of the azimuthal decorrelation using a normalised dijet cross section were reported by CMS and D0 collaborations \cite{ref:D0Azi,ref:CMSAzi}.
%Other Experiments:
% D0 :  ref:D0Azi
% CMS: ref:CMSAzi
% CDF: 

