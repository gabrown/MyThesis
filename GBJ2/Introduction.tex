The first LHC measurement of dijet production with a jet veto (documented in Chapter \ref{chp:GBJ1} and \cite{ref:ATLASGap}) examined the gap fraction in slices \ptb{} up to $\dy{}=6$.
This cut-off in \dy{} was necessary because the analysis used a central jet trigger strategy in which the trigger fired if a jet was within $|y|<2.8$.
There was a relatively small number of events at large \dy{} and where the statistical uncertainties were large. 
The data was compared to state of the art theory predictions by HEJ and POWHEG.
Whilst the HEJ curves described the with data as a function of \dy{}, the POWHEG curves did not.

The analysis was extended to probe the gap fraction at large \dy{} region.
The large \dy{} region is again studied using the 2010 data, but using a new dijet trigger strategy that allowed measurements up to $\dy{}=8$, which is the limit of the detector.
In addition, the effect of quark/gluon emission from the dijet system is studied using the azimuthal decorrelation between the jets that make up the dijet system.
The previous ATLAS measurement of azimuthal decorrelation, \cite{ref:Decorr}, was carried out using jets with an $\eta{} < 0.8$.

