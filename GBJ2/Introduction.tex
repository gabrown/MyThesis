The first LHC measurement of dijet production with a jet veto (documented in Chapter \ref{sec:GBJ1}) examined the gap fraction in slices of both \dy{} and \ptb{}, up to $\dy{}=6$.
This cut-off in \dy{} was necessary because the analysis had a central jet trigger strategy in which the trigger fired if a jet was within $|y|<2.8$.
There was a low number of events at larger \dy{} regions and the systematic and statistical uncertainties was large. 
Whilst the HEJ curves agreed well with data as a function of \dy{}, the POWHEG curves did not.

The previous analysis was extended in two ways to probe the large \dy{} region and the high \ptb{} region.
The large \dy{} region was studied using the 2010 data with a dijet trigger strategy that allowed measurements upto $\dy{}=8$.
The large \ptb{} region was studied using the 2011 data using a central jet trigger strategy with measurements upto $\ptb{}=1$ TeV.
The study into the large \dy{} region is presented in this chapter.

In addition, quark/gluon emission from the dijet system is studied using the azimuthal decorralation of the dijet pair.
The previous ATLAS measurement of azimuthal decorralation, \cite{ref:Decorr}, was carried out using jets with an $\eta{} < 0.8$.
The new analysis looks at variable azimuthal decorralation observables for inclusive events and those events that survive a central jet veto.


